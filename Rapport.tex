\documentclass[11pt, a4paper]{article}

\usepackage[utf8]{inputenc} 	% L'encodage. Ca peut être à remplacer par \usepackage[utf8]{inputenc}
\usepackage[T1]{fontenc}
\def\ques{\noindent{\bf Question : }}
\usepackage[english]{babel}
\usepackage[top=2cm, bottom=2cm, left=2cm, right=2cm]{geometry}		% Les marges
%\usepackage{listings}		% Pour mettre du code dans le fichier, il suffira ensuite d'appeler \begin{lstlisting}
%\lstset{
%language=Python,        	% choix du langage : C, Python, R, PHP...
%basicstyle=\footnotesize\color{mygray},       % taille de la police du code
%numbers=left,                   % placer les num�ros de lignes à droite (right) ou à gauche (left)
%numberstyle=\footnotesize,        % taille de la police des num�ros
%numbersep=7pt,                  % distance entre le code et sa num�rotation
%stepnumber=1,
%tabsize=4
%}
\usepackage{lscape}
\usepackage{amsthm}
\newtheorem{theorem}{Theorem}
\newtheorem{definition}{Definition}
\setcounter{theorem}{0}
\usepackage{comment} 
\usepackage{float}
%\usepackage{capt-of}
\usepackage{graphicx}	% Si besoin de bosser sur des images. Si besoin, aller sur le SDZ.
\usepackage{color}
\definecolor{mygray}{rgb}{0.2,0.2,0.2}
\usepackage{xcolor}
\usepackage{multicol}
% Les packages pour mettre des expressions un peu math�matis�es.
\usepackage{amsmath}
\usepackage{amssymb}
\usepackage{amsthm}
\usepackage{mathrsfs}
%\usepackage{mathtools}
%\usepackage{dsfont}
\newcommand{\horrule}[1]{\rule{\linewidth}{#1}}
\newcommand{\SR}[2]{\textcolor{gray}{#1}\textcolor{blue}{#2}}


\newtheorem{question}{Q}
\title{	
Composite likelihood inference of the parameters of a Poisson log-normale distribution
}

\date{\today}
\author{Elsa TEULIERE}

\graphicspath{ {./figures/} }
 

\begin{document}
\maketitle
\vspace{1cm}
\section*{Introduction}
\section{Poisson log-normal model}
The Poisson log normal model was developped by Aitchison \textit{et al.}\cite{aitchison1989multivariate}, to modell multivariate count data.  The idea is to introduce latent gaussian variables encoding the dependency between the variables. In this section we introduce this model and our notation and briefly review the existing litterature on this model.
\subsection{Definitions and notations}
We note $\mathcal{N}(p,Q)$ the multivariate normale distribution of mean $p$ and variance-covariance $Q$, and $\mathcal{P}(\alpha)$ the Poisson-distribution of mean $\alpha$.
\subsubsection{The Model} 
We consider $(\mu,\Sigma) \in \mathbb{R}^n \times \mathcal{M}_n(\mathbb{R})$, with $\Sigma$ a symetric non negative matrix, the parameters of the model.\\
We consider the latent variables $Z \sim \mathcal{N}(0,\Sigma)$.\\
Multivariate count data $(Y_1,...,Y_n)$ follow a Poisson log-normal model of mean $\mu$ and matrix of variance-covariance $\Sigma$ if :
\begin{center}
$\forall i \in \{1,...n\}, Y_i \mid Z_i \sim \mathcal{P}(e^{\mu_i+Z_j})$
\end{center}
and 
\begin{center}
$\forall (i,j) \in \{1,...,n\}^2$, $ Y_i\mid Z_i$ and $Y_j\mid Z_j$ independant.
\end{center}

We typically consider $N$ independant sampling experiment of $n$ variables each time, for example the case of sampling $n$ species over $N$ sites. In our notations, the exposant allways stands for the sampling experiment, and the underscore stand for the observed variable.\\
So we now have $(Z^j)_{j \in \mathbb{N}}$ latent variables, independant and identically distributed : $\forall j \in \{1,...,N\}$, $Z^j \sim \mathcal{N}(0,\Sigma)$, and :
\begin{center}
$\forall (i,j) \in \{1,...,n\} \times \{1,...,N\}$, $Y^j_i \mid Z^j_i \sim \mathcal{P}(e^{\mu_i+Z^j_i})$.
\end{center}
The advantage of this model is that all the dependance structure is encoded through gaussian variables.
\subsubsection{Extension accounting for covariates and offset}
In multivariate data, the dependency can occur through covariates. As developped in \cite{chiquet2017variational}, we will postulate the existence of a linear regression in  the parameters space. \\
\\
In this case, we consider a vector $X^j \in \mathbb{R}^d$ of covariates (collected for each sampling). Let $M$ be a $d \times n$ Matrix, called the matrix of regression parameters. We note $\mu^i$ the $i^{th}$ column of the matrix M. We can also add one offset parameter per observation $O^j_i$. The offset, can typically be the sampling effort. In this case the Poisson log-normal model is given by :
\begin{center}
$\forall (i,j) \in \{1,...,n\} \times \{1,...,N\}$, $Y^j_i \mid Z^j_i \sim \mathcal{P}(e^{O^j_i + X_j^T \mu^i +Z^j_i})$ 
\end{center}
The set of parameters, that we aim to estimate is now given by $(M,\Sigma) \in \mathcal{M}_{d \times n}(\mathbb{R}) \times \mathcal{M}_{n}(\mathbb{R})$. $M$ is the matrix of regression parameter. Each column of $M$ contain all the regression parameters toward the covariables for one observed variable. We will now use the following notations : $X \in \mathcal{M}_{d \times N} ( \mathbb{R})$ stands for the matrix of covariates. The $j^{th}$ column of $X$ contains the covariates for the $j^{th}$ sampling. $O =(O_{ij}) \in \mathcal{M}_{n \times N} (\mathbb{R})$ stands for the matrix of offsets, we consider that $O_{ij}=O^j_i$. \\
\\
This model enable us to take in account for fixed additional effects. Indeed, it can help to interpret the dependency parameters. For examples, 	Poisson log-normal model can be used to estimate an interaction network from the co-occurancy. With this model, we can hope, with a good choice of the covariables, that the dependencies encoded in the variance-covariance matrix $\Sigma$ will indicate the interraction between species.
\subsection{Properties}
\subsubsection{Density function of the Poisson log-normal distribution}
Following \cite{aitchison1989multivariate}, the Poisson log-normale distribution admits a density function given by :
\begin{center}
$\forall (m_1,...m_n) \in \mathbb{N}^n$, $h_{(\mu,\Sigma)}(m_1,...m_n)=\int_{\mathbb{R}^n} \prod_{i=1}^n f_{e^{\mu_i+z_i}}(m_i) g_{(0,\Sigma)}(z_1,...z_n) \mathrm{d}z_1...\mathrm{d}z_n$
\end{center}
Where $f_{\alpha}$ stands for the density function of a Poisson distribution with parameter $\alpha$, and $ g_{(0,\Sigma)}$ for the density function of a multivariate Normal distribution with parameters $(0,\Sigma)$.\\
\\
Taking in account covariates and offset, give the following distribution function :
\begin{center}
$ \forall X \in \mathbb{R}^d, \forall O \in \mathbb{R}^d, \forall (m_1,...m_n) \in \mathbb{N}^n$, $h_{(M,\Sigma \mid X, O)}(m_1,...m_n)=\int_{\mathbb{R}^n} \prod_{i=1}^n f_{e^{O_i+X^T \mu^i+z_i}}(m_i) g_{(0,\Sigma)}(z_1,...z_n) \mathrm{d}z_1...\mathrm{d}z_n$
\end{center}
\subsubsection{Moments of the Poisson log-normale distribution} Again, following \cite{aitchison1989multivariate}, the moments of the Poisson log-normale distribution can easily be obtained through conditional expectation. Considering the observation $Y^j_i$ is following a Poisson log-normal distribution with  parameters $(M , \Sigma)$ (with $\Sigma= (\sigma_{ij})$), the covariates $X \in \mathcal{M}_{d \times N} ( \mathbb{R})$ and the offset $O \in \mathcal{M}_{n \times N} (\mathbb{R})$ 
\begin{align*}
\mathbb{E}(Y^j_i)& =\mathrm{exp}(O_{ij}+ (X^j)^T\mu^i+\frac{1}{2}\sigma_{ii})\\
\mathbb{V}(Y^j_i)& =\mathrm{exp}(O_{ij}+(X^j)^T \mu^i+\frac{1}{2}\sigma_{ii})+(\mathrm{exp}(O_{ij}+(X^j)^T\mu^i+\frac{1}{2}\sigma_{ii})^2(\mathrm{exp}(\sigma_{ii})-1)\\
\mathrm{Cov}(Y^j_i,Y^j_k)& =\mathrm{exp}(O_{ij}+(X^j)^T \mu^i+\frac{1}{2}\sigma_{ii})\mathrm{exp}(O_{kj}+(X^j)^T \mu_k+\frac{1}{2}\sigma_{kk})(exp(\sigma_{ik}) - 1)
\end{align*}
\subsubsection{Overdispersion}
From the calculation of the moments, we see that for $Y^j_i$ following a Poisson log-normal distribution :
\begin{center}
$\mathbb{E}(Y^j_i) < \mathbb{V}(Y^j_i)$.   
\end{center}
The Poisson log-normal distribution is overdispersed, giving a clue that it can be applyed to a large range of multivariate count data, specially in ecology.
Indeed $\mathrm{Cov}(Y^j_i,Y^j_k)$ and $\sigma_{ik}$ have the same sign.
\subsection{Variational estimation of the parameters}
\textit{\`a travailler PLN/PCA}
\SR{}{Pour ce passage, il faut faire court en rappelant le principe de VEM et en disant qu'on sait le faire pour PLN.}
\section{Composite likelihood estimation}
\subsection{Definition and notation}
\subsubsection{Presentation of the composite likelihood}
One classical method to estimate the parameters of a distribution is to maximize a so called likelihood-function over all the parameters sets. The likelihood-function is typically the density of the modelled distribution taken over all observations :
\begin{center}
$\mathcal{L}_{(Y^1,..Y^N)}(M,\Sigma \mid X,O) = \prod_{j=1}^N h_{(M,\Sigma \mid X,O)}(Y^j)$
\end{center}
When the considered density function is in the exponential family, people often take the log-likelihood, expressed as follow : 
$\mathcal{L}_{(Y^1,..Y^N)}(M,\Sigma \mid X,O) = \sum_{j=1}^N \mathrm{log} (h_{(M,\Sigma \mid X,O)}(Y^j))$.\\
\\ 
In our case, the likelihood function is costly to calculate, because the dependency structure in the data implies to calculate integrals over $\mathbb{R}^n$. Several approached have been proposed in order to symplify the likelyhood function in the case of complex dependencies. We propose to use the composite likelihood approach (see\cite{varin2011overview}, \cite{pedeli2018pairwise} and \cite{varin2008composite}).\\
\\
Composite likelihood is expressed as a weighted product of the marginal or conditional densities.
\begin{definition}\cite{varin2011overview} Let $Y$ be a $n$-dimensional random vector, with density function $f_\theta$ parametrized by a $p$-dimensional unknown parameter $\theta \in \Theta$.\\
Let $\{\mathcal{A}_1,..., \mathcal{A}_k\}$ be a set of marginal or conditional events. We note $\mathcal{L}^k_{(y)} (\theta) \varpropto f_\theta(y)$ the associated (marginal or conditional) likelihood. The composite likelihood is then given by :
\begin{align*}
\mathcal{CL}_{(y)}(\theta) = \prod_{i=1}^k (\mathcal{L}^i_{(y)}(\theta))^{w_i} 
\end{align*}
Where $w_i$ are non negative weights to be choosen.
\end{definition}
\\
The idea behind composite marginal likelihood \cite{varin2008} is to compose low dimensional marginal densities in order to symplify the calculations, but also to capture the dependence between the parameters. \\
If there is no dependance structure, it is sufficient to take the product of the one-dimensional densities. Otherwise, the two-dimensionnal marginal densities are needed at least, to capture the  dependence between the parameters.\\
\\
No theoretical results exist about the loss of efficiency \cite{lele2006sampling}. The idea is to show that maximizing the composite likelihood is a consitent assymptotic estimator of the parameters.

\subsubsection{Composite likelihood for the PLN model}
In our case, by integration, we can show that the two-dimensional marginal density function is given by $ \forall (m_1,m_2) \in \mathbb{N}^2$,$\forall X \in \mathbb{R}^d$, $ \forall O \in \mathbb{R}^d$ :
\begin{center}
 $\sum_{(m_3...m_n) \in \mathbb{N}^{n-2}} h_{(\mu,\Sigma \mid X)}(m_1,...m_n) = \int_{\mathbb{R}^2} f_{e^{O_1+X^T \mu^1+z_1}}(m_1) f_{e^{O_2+X^T \mu^2+z_2}}(m_2) g_{(0,\Sigma^{(12)})}(z_1,z_2)\mathrm{d}z_1 \mathrm{d}z_2$.
\end{center}
with $\Sigma^{(12)}=\begin{pmatrix}
\sigma_{11} & \sigma_{12} \\
\sigma_{12} & \sigma_{22}\\
\end{pmatrix}$\\
With this simple expression of the pairwise density function, we can write the composite pairwise marginal density function given by : $\forall (Y^1,...,Y^N) \in (\mathbb{R}^n)^N$, $\forall O \in \mathcal{M}_{n \times d} (\mathbb{R})$, $\forall X \in \mathcal{M}_{d \times N}(\mathbb{R})$ :
\begin{align}
\mathcal{CL}_{(Y^1,...,Y^N,O)}(M,\Sigma \mid X) = \prod_{j=1}^N \prod_{1 \leq i < k \leq n}  h'_{(M^{(ik)},\Sigma^{(ik)} \mid X^j,O^j)}(Y^j_i,Y^j_k)
\end{align}
With $h'$ the pairwise density function defined as above , $M^{(ik)}$ the matrix constituted of the $i$-th and $k$-th row of M and 
$\Sigma^{(ik)}  = \begin{pmatrix}
\sigma_{ii} & \sigma_{ik}\\
\sigma_{ik} & \sigma_{kk}\\
\end{pmatrix}$
For easier calculation, we take the log composite likelihood given by :
\begin{align*}
\mathcal{CL}_{(Y^1,...,Y^N,O)}(M,\Sigma \mid X) = \sum_{j=1}^N \sum_{1 \leq i < k \leq n}  \mathrm{log}(h'_{(M^{(ik)},\Sigma^{(ik)} \mid X^j,O^j)}(Y^j_i,Y^j_k))
\end{align*}
There exists some theoretical results about the consistency of the method of maximisation of the composite likelihood. In the next section we show that the pariwise log-composite likelihood, is an M-estimator and so all the theory developped for M-estimators \cite{vaart_1998} can be applied.
\subsubsection{M-estimator and composite likelihood}
\begin{definition}
Let $(Y^1,...Y^N) \in \mathcal{X}^N$ be a set of observations. Let $\theta$ be an unknown parameter. \\
$\widehat{\theta}_N(Y^1,...Y^N)$ is called an M-estimator of $\theta$ if it maximize a function :
\begin{align*}
\mathcal{M}_N : \theta \mapsto \frac{1}{N} \sum_{i=1}^N m_\theta(Y^i)
\end{align*}
with, for all $\theta \in \Theta$, $m_\theta : \mathcal{X} \rightarrow \mathbb{R}$ a known function.
\end{definition}
Since the $\frac{1}{N}$ factor in the M-estimator doesn't modify the location of the maximum, we can also define the M-estimator as the parameter miximizing a function $\mathcal{M}_N : \theta \mapsto  \sum_{i=1}^N m_\theta(Y^i)$.\\
\\
In the case of the composite likelihood, for all $\theta \in \Theta$, we set 
\begin{center}
$\begin{array}{ccccc}
m_\theta & : & \mathcal{X} & \to & \mathbb{R} \\
 & & Y^j & \mapsto & \sum_{0 \leq i<k \leq n} \mathrm{log} (h'_{(M^{(ik)},\Sigma^{(ik)}}(Y^j_i,Y^j_k) \\
\end{array}$
\end{center}
The estimator constructed by maximizing the pairwise composite likelihodd is an M-estimator. We will apply the theory developped for M-estimators to prove the consistency and assymptotic normality of the estimator.
\subsection{Consistency}
\subsubsection{Consistency of M-estimators}
\begin{theorem} \label{ThMest} \cite{vaart_1998}
Let $(\mathrm{M}_N)$ be a random sequence of functions in the variable $\theta$ and $M$ a determinist function in the variable $\theta$. If :
\begin{enumerate}
\item $\underset{\theta \in \Theta}{\mathrm{sup}} \mid \mathrm{M}_N(\theta)-\mathrm{M}(\theta) \mid \overset{\mathbb{P}}{\longrightarrow} 0$
\item the maximum $\theta^\ast$ of M is unique.
\end{enumerate}
Then any sequence of estimators $\widehat{\theta}_N$ with $\mathrm{M}_N(\widehat{\theta}_N) \geq \mathrm{M}_N(\theta^\ast)-\circ_p(1)$ converges in probability to $\theta^\ast$.
\end{theorem}
\begin{proof}
Let $M_N$ a random sequence of function, $M$ a determinist function with a unique maximum $\theta^*$ and $\widehat{\theta}_N$ be a sequence of estimators satisfying the conditions of the theorem \ref{ThMest}.

We want to proove that :
\begin{center}
$\forall \epsilon > 0$, $\mathbb{P}(\{\mathrm{d}(\widehat{\theta}_N,\theta*)>\epsilon\})\underset{ N \rightarrow + \infty}{\longrightarrow} 0$
\end{center}
Let consider $\epsilon >0$. By assumption, $\theta ^*$ is the unique maximum of the function $M$, so $\underset{ \theta : \mathrm{d}(\theta,\theta^*)}{\mathrm{sup}}(M(\theta)) < M(\theta^*)$. In particular, there exists $\eta > 0$ so that : $\forall \theta \in \Theta$, $\mathrm{d}(\theta,\theta^*)>\epsilon$, $M(\theta) < M(\theta^*)-\eta$.\\
Consequently, the probabilist event $\{\mathrm{d}(\widehat{\theta}_N,\theta^*)> \epsilon\}$ is included in the event $\{M(\widehat{\theta}_N)<M(\theta^*)-\eta\}$. We will show that the probability of this event converges to zero when $N$ tends to infinity.\\
The assumption of uniform convergence of the sequence of functions $M_N$  to $M$ ensures that $M_N(\theta^*) \overset{\mathbb{P}}{\longrightarrow} M(\theta^*)$. Indeed, by assumption $M_N( \widehat{\theta}_N) \geq M_N(\theta^*)- \circ_\mathbb{P}(1)$, so we deduce that $M_N( \widehat{\theta}_N)+ \circ_\mathbb{P}(1) \geq M(\theta^*)$.
We now deduce that :
\begin{align*}
M(\theta^*)-M(\widehat{\theta}_N) & \leq \widehat{\theta}_N)+ \circ_\mathbb{P}(1)-M(\widehat{\theta}_N)\\
& \leq \underset{\theta}{\mathrm{sup}} \mid M_N-M \mid (\theta) + \circ_\mathbb{P}(1)
\end{align*}
This is sufficient to conclude since for all $\eta > 0$, $\mathbb{P} ( \mid M(\theta^*)-M(\widehat{\theta}_N)\mid > \eta) \underset{ N \rightarrow + \infty} {\longrightarrow} 0$.
\end{proof}
\subsubsection{Consistency of the composite likelihood for the PLN model}
The estimator of the parameters $M$ and $\Sigma$ of a Poisson log-normale model constructed by maximizing the composite likelihood is consistent. We show this result in two steps. We first show that for a Poisson log-normal model in dimesion two, the maximum-likelihood estimator is consistent before generalizing to $n$-dimensional PLN models.

\paragraph{In dimension 2} :
\begin{theorem} \label{Consistence_2}
For each $(i,k) \in \{1...n\}^2$, $i \neq k$, the estimator $(\widehat{M}^{(ik)}_N,\widehat{\Sigma}^{(ik)}_N)$ constructed by maximizing the log-likelihood of the couple $(Y^j_i,Y^j_k)_{j \in \{1...N\}}$, given by $\sum_{i=1}^{N} \mathrm{log}(p_{(M^{(ik)},\Sigma^{ik} \mid X)}(Y^j_i,Y^j_k))$, is a consistent estimator of the correlation coefficients $M^{(ik)}$ and of the variance-covariance matrix $\Sigma^{(ik)}$.
\end{theorem}
\begin{proof}

The idea of this proof is to use the theorem \ref{ThMest} for our composite likelihood. The proof will  follow four steps to verify that the assumptions of the theorem are verified. We will first show the convergence in probability of our M-estimator to a function $\mathrm{M}_{(M^{ik},\Sigma^{ik})}$. Then we proove the existence of a unique maximum for this function $\mathrm{M}$. To do this we need two steps : at first we show that our model is identifiable and then, using the Kullback-divergence, that it has a maximum. Combining this two steps will allow us to conclude the existence of a unique maximum. In the last part, we conclude, using the theroem \ref{ThMest} on the consistence of the estimator. \\
\\
\textbf{Step 1 :} Convergence in probability.\\
Using the large number low, we do have that $\frac{1}{N} \sum_{i=1}^{N} \mathrm{log}(h_{(M^{(ik)},\Sigma^{(ik)} \mid X)}(Y^j_i,Y^j_k))$ converge almost surely to $\mathbb{E}_X[\mathbb{E}_{Y_i Y_k \mid X}[ \mathrm{log}(h_{(M^{(ik)},\Sigma^{(ik)} \mid X)}(Y_i,Y_k))]]$. So we set $\mathrm{M}(M^{(ik)}, \Sigma^{(ik)}) =\mathbb{E}_X[\mathbb{E}_{Y_i Y_k \mid X}[ \mathrm{log}(h_{(M^{(ik)},\Sigma^{(ik)} \mid X)}(Y_i,Y_k))]]$. \\
\\
\textbf{Step 2 :} Identifiability of the model.\\
 We use the general definition of identifiability \cite{rivoirard}, namely that the family of probabilities $\mathcal{P}=\{\mathbb{P}_\theta,\theta \in \Theta\}$ is identifiable if the application $ \theta \mapsto \mathbb{P}_\theta$ is injective.
 In our case $\mathcal{P}=\{(\mathcal{PLN}_X (M,\Sigma))_{X \in \mathcal{X}}; (\mu, \Sigma) \in  \mathcal{M}_{d \times n}(\mathbb{R}) \times \mathcal{M}_n(\mathbb{R})\}$. To show the identifiability of the model we will use the fact that two variables having the same distribution have the same moment. Consider $(M^{(ik)},\Sigma^{(ik)}) \in \mathcal{M}_{d\times 2}(\mathbb{R}) \times \mathcal{M}_2(\mathbb{R})$ and $(M'^{(ik)},\Sigma'^{(ik)}) \in \mathcal{M}_{d\times 2}(\mathbb{R}) \times \mathcal{M}_2(\mathbb{R})$, so that  for every vector of covariables $X \in \mathcal{X}$,$\mathcal{PLN}_X(M^{(ik)},\Sigma^{(ik)})\sim\mathcal{PLN}_X(M'^{(ik)},\Sigma'^{(ik)})$ . We have :
 \begin{equation}
 \mathbb{E}_{(M^{(ik)},\Sigma^{(ik)} \mid X)}[Y_i]=\mathbb{E}_{(M'^{(ik)},\Sigma'^{(ik)} \mid X)} [Y_i] 
 \end{equation}
\begin{equation}
 \mathrm{Var}_{(M^{(ik)},\Sigma^{(ik)} \mid X)}[Y_i]=\mathrm{Var}_{(M'^{(ik)},\Sigma'^{(ik)} \mid X)} [Y_i] \\
\end{equation}
\begin{equation}
 \mathrm{Cov}_{(\mu^{(ik)},\Sigma^{(ik)} \mid X)}[Y_i,Y_k]=\mathrm{Cov}_{(M'^{(ik)},\Sigma'^{(ik)} \mid X)} [Y_i,Y_k]
\end{equation}
Using the formula of this moment we recall in the presentation of the model, we find that $\sigma_{ii} = \sigma'_{ii}$, $\sigma_{ik}=\sigma'_{ik}$ and $X (M^i-M'^i)=0$. The last condition has to be true for any vector of covariable $X$. We deduce from this that,  we expect $\mathcal{X}^\perp = \{0\}$ for the model to be identifable, \textit{ie.} that $\mathrm{rg}(\mathcal{X})=d$.\\
\\
\textbf{Step 3 :} Existence of a unique maximum for the function $\mathrm{M}$.\\
What we consider is nothing else than the log-likelihood of two variables with a Poisson-log-normale distribution. We just shew that the model is identifiable. So there exist one and only one set of parameters $(M^{(ik)}^\ast, \Sigma^{(ik)}^\ast)$ such that $(Y_j,Y_k) \sim \mathcal{PLN}_X(M^{(ik)}^\ast,\Sigma_{(ik)}^\ast)$. We first can show that maximizing the log-likelihood is equivalent to minimize the Kullback divergence. We recall that the Kullback divergence for two distributions of density functions $p_\theta$ a,d $p_{\theta^\ast}$ is given by $\mathrm{D_{KL}}(p_{\theta^\ast} \parallel p_{\theta})= \int p_{\theta^\ast}(x) \mathrm{log}(\frac{p_{\theta^\ast}(x)}{p_{\theta}(x)}) \mathrm{d}x$. Since we have :
\begin{align*}
\mathbb{E}_X[\mathbb{E}_{(M^{(ik)}^\ast,\Sigma^{(ik)}^\ast \mid X)}&[\mathrm{log}(h_{(M^{(ik)}^\ast,\Sigma^{(ik)}^\ast \mid X)}(Y_i,Y_k)) - \mathrm{log}(h_{(M^{(ik)},\Sigma^{(ik)} \mid X)}(Y_j,Y_k))]] \\
& = \mathbb{E}_X[\mathrm{D_{KL}}(h_{(\mu_{jk}^\ast,\Sigma_{jk}^\ast \mid X)}\parallel h_{(\mu^{(ik)},\Sigma^{(ik)} \mid X)})]\\
& = \mathbb{E}_X[\mathbb{E}_{(M^{(ik)}^\ast,\Sigma^{(ik)}^\ast \mid X)}[\mathrm{log}(h_{(M^{(ik)}^\ast,\Sigma^{(ik)}^\ast \mid X)}(Y_i,Y_k))]] - \mathrm{M}(M^{(ik)},\Sigma^{(ik)})
\end{align*}
 Since the Kullback-divergence is always positive, we see that maximizing the log-likelihood is equivalent to minimize the Kullback-divergence. The Kullback-divergence is only zero if the two distributions are the same. So we see that the M function is maximum for $(M^{(ik)},\Sigma^{(ik)})=(M^{(ik)}^\ast,\Sigma^{(ik)}^\ast)$. So we can conclude that the function M only has one maximum, and this maximum is obtained for the parameters we are interested in.\\
\\
\textbf{Step 4 :} Conclusion.\\
To summurize we have :
\begin{enumerate}
\item By the convergence almost surely :
\begin{center}
 $\underset{(M^{(ik)},\Sigma^{(ik)})}{\mathrm{sup}} \mid \frac{1}{N} \sum_{j=1}^{N} \mathrm{log}(h_{(M^{(ik)},\Sigma^{(ik)} \mid X)}(Y^j_i,Y^j_k)) - \mathbb{E}_X[\mathbb{E}_{Y_i Y_k \mid X}[\mathrm{log}(h_{(M^{(ik)},\Sigma^{(ik)} \mid X)}(Y_i,Y_k))]] \mid \overset{\mathbb{P}}{\longrightarrow} 0 $ 
 \end{center}
\item The function $(M^{(ik)},\Sigma^{(ik)})\mapsto \mathbb{E}_X[\mathbb{E}_{Y_i Y_k \mid X}[\mathrm{log}(h_{(M^{(ik)},\Sigma^{(ik)} \mid X)}(Y_i,Y_k))]] $ only has one maximum for $(M^{ik},\Sigma^{(ik)})=(M^{(ik)}^\ast,\Sigma^{(ik)}^\ast)$, the parameters of the Poisson-log-Normale distribution of the variables $(Y_i,Y_k)$.
\end{enumerate}
Applying theorem \ref{ThMest}, we conclude that the sequence of estimators $(\widehat{M}^{(ik)}_N, \widehat{\Sigma}^{ik}_N)_{N \in \mathbb{N}}$ converges in probability to what we hope to approximate, namely $(M^{(ik)}^\ast, \Sigma^{(ik)}^\ast$).
\end{proof}

\paragraph{In dimension $n$}
\begin{theorem}
The estimator $(\widehat{M}_N,\widehat{\Sigma}_N)$ constructed by maximizing the composite pairwise likelihood for $(Y^j_1,...Y^j_n)_{j \in \{1...N\}}$ is a consistent estimator of the correlation coefficients $M$ and of the matrix of variance-covariance $\Sigma$.
\end{theorem}

\begin{proof}
The proof follows exactly the same path that we did for a couple of variables. We will again use theorem \ref{ThMest} and the results of theorem \ref{Consistence_2}.\\
\\
\textbf{Step 1 :} Convergence in probability.
Again using the large number law we have :\\
 $\frac{1}{N}\sum_{i=1}^{N} \sum_{j<k} log(p_{(M^{(ik)},\Sigma^{(ik)} \mid X)} (Y^j_i,Y^j_k) \overset{\mathbb{P}}{\longrightarrow} \sum_{j<k}\mathbb{E}_X [\mathbb{E}_{Y_i Y_k \mid X}[\mathrm{log}(p_{(M^{(ik)},\Sigma^{(ik)} \mid X)}(Y_i,Y_k))]]$.\\
 We note M the function $\mathrm{M} : (M, \Sigma) \mapsto \sum_{j<k}\mathbb{E}_X [\mathbb{E}_{Y_i Y_k \mid X}[\mathrm{log}(p_{(M^{(ik)},\Sigma^{(ik)} \mid X)}(Y_i,Y_k))]]$.\\
 \\
 \textbf{Step 2 :} Identifiability of the model.\\
The family of model we consider here is $\mathcal{P}=\{ (\mathcal{PLN}_X(\mu,\Sigma))_{X \in \mathcal{X}}; M \in \mathcal{M}_{d \times n}(\mathbb{R}), \Sigma \in \mathcal{M}_n(\mathbb{R})\}$. Using the moment as we did with a Poisson log-normale distribution of a vector of two variables, but this time for n variables is sufficient to show the identifiability of the model.\\
\\
\textbf{Step 3 :} Existence and uniqueness of the maximum for the function M.\\
Again here we have by the identifiability of the model, under the assumption that the data follow a Poisson log-normale distribution, that there exist one and only one set of parameters $(M^\ast, \Sigma^\ast)$ such as given a vector of covariables $X$, $(Y^j) \sim \mathcal{PLN}_X (M^\ast, \Sigma^\ast)$. As we did in the previous section we show that maximizing M is equivalent to minimize the function :\\
$(M, \Sigma) \mapsto \sum_{j<k} \mathbb{E}_X [\mathrm{D_{KL}}(p_{(M^{(ik)}^\ast,\Sigma^{(ik)}^\ast \mid X)} \parallel p_{(M^{(ik)},\Sigma^{(ik)} \mid X)}]$.\\
We have here a finite sum of positives variables, so in order to minimize it, we can minimize each of the terms. So the above function is minimal for $M^\ast=(M^{(i)}^\ast)_{i \in \{1...n\}}$ and $\Sigma^\ast$ defined as follow : the term $(\sigma_{ik})_{i \neq k}$ of $\Sigma^\ast$ is the term $\sigma_{12}$ of $\Sigma^{(ik)}^\ast$  and the term $\sigma_{ii}$ of $\Sigma^\ast$ is the term $\sigma_{11}$ of $\Sigma^{(ik)}^\ast$.\\
Thus we have the existence of a maximum and its uniqueness.\\
\\
\textbf{Step 4 :} Conclusion.\\
Applying theorem \ref{ThMest}, we conclude that the sequence of estimators $(\widehat{M}_N, \widehat{\Sigma}_N)_{N \in \mathbb{N}}$ maximizing the composite likelihood function $(M,\Sigma) \mapsto \mathcal{CL}_{(M,\Sigma) \mid X}((Y^j)_{j \in \{1..N\}})$  converges in probability to what we hope to approximate, namely $(M^\ast, \Sigma^\ast)$, the parameters of our Poisson log-normale model.
\end{proof}
\subsubsection{Estimation of the maximum.}
In order to estimate the maximum of the composite pairwise likelihood we performed an optimization algorithme with the gradients. We calculate the gradients and performed simulations to estimate the parameters.\\
The calculation of the density function of the Poisson log-normale distribution are quiet costly to calculate. Since the optimization require calculation of the density function, we propose to initialise the optimisation algorithme with the results of the VEM for the Poisson log-normal model.
\subsection{Assymptotic normality}
The assymptotic normality is needed to construct tests in order to construct assymptotic tests to estimate the parameters and get assymptotic confidence intervalls.
\subsubsection{M-estimator theory}

\subsubsection{Assymptotic normality of the composite likelihood for the PLN model}
\textit{proof}


\section{Composite likelihood inference for spatial data}
\subsection{Poisson log-normal model for spatial data}
\subsubsection{One specie\SR{}{s} \SR{(meme au singulier)}{}, spatial dependency}
\subsubsection{spatial parametrisation of the PLN}
\subsection{Computational optimisation by taking a sparce CL}

\section{Illustrations}
\subsection{Simulations}
\subsection{'Tiques' data}

\appendix
\section{Appendix}
\subsection{Gradients}
\subsection{Hessian}
\subsection{Spatial gradient and hessian}


\bibliographystyle{plain}
\bibliography{Biblio.bib}
\end{document}