\documentclass[11pt, a4paper]{article}

\usepackage[utf8]{inputenc} 	% L'encodage. Ca peut être à remplacer par \usepackage[utf8]{inputenc}
\usepackage[T1]{fontenc}
\def\ques{\noindent{\bf Question : }}
\usepackage[english]{babel}
\usepackage[top=2cm, bottom=2cm, left=2cm, right=2cm]{geometry}		% Les marges
%\usepackage{listings}		% Pour mettre du code dans le fichier, il suffira ensuite d'appeler \begin{lstlisting}
%\lstset{
%language=Python,        	% choix du langage : C, Python, R, PHP...
%basicstyle=\footnotesize\color{mygray},       % taille de la police du code
%numbers=left,                   % placer les num�ros de lignes à droite (right) ou à gauche (left)
%numberstyle=\footnotesize,        % taille de la police des num�ros
%numbersep=7pt,                  % distance entre le code et sa num�rotation
%stepnumber=1,
%tabsize=4
%}
\usepackage{lscape}
\usepackage{amsthm}
\newtheorem{theorem}{Theorem}
\newtheorem{definition}{Definition}
\setcounter{theorem}{0}
\usepackage{comment} 
\usepackage{float}
%\usepackage{capt-of}
\usepackage{graphicx}	% Si besoin de bosser sur des images. Si besoin, aller sur le SDZ.
\usepackage{color}
\definecolor{mygray}{rgb}{0.2,0.2,0.2}
\usepackage{xcolor}
\usepackage{multicol}
% Les packages pour mettre des expressions un peu math�matis�es.
\usepackage{amsmath}
\usepackage{amssymb}
\usepackage{amsthm}
\usepackage{mathrsfs}
%\usepackage{mathtools}
%\usepackage{dsfont}
\newcommand{\horrule}[1]{\rule{\linewidth}{#1}}
\newcommand{\SR}[2]{\textcolor{gray}{#1}\textcolor{blue}{#2}}


\newtheorem{question}{Q}
\title{	
Composite likelihood inference of the parameters of a Poisson log-normale distribution
}

\date{\today}
\author{Elsa TEULIERE}

\graphicspath{ {./figures/} }
 

\begin{document}
\maketitle
\vspace{1cm}
\section*{Introduction}
\section{Poisson log-normal model}
The Poisson log normal model was developped by Aitchison \textit{et al.}\cite{aitchison1989multivariate}, to modell multivariate count data.  The idea is to introduce latent gaussian variables encoding the dependency between the variables. In this section we introduce this model and our notation and briefly review the existing litterature on this model.
\subsection{Definitions and notations}
We note $\mathcal{N}(p,Q)$ the multivariate normale distribution of mean $p$ and variance-covariance $Q$, and $\mathcal{P}(\alpha)$ the Poisson-distribution of mean $\alpha$.
\subsubsection{The Model} 
We consider $(\mu,\Sigma) \in \mathbb{R}^n \times \mathcal{M}_n(\mathbb{R})$, with $\Sigma$ a symetric non negative matrix, the parameters of the model.\\
We consider the latent variables $Z \sim \mathcal{N}(0,\Sigma)$.\\
Multivariate count data $(Y_1,...,Y_n)$ follow a Poisson log-normal model of mean $\mu$ and matrix of variance-covariance $\Sigma$ if :
\begin{center}
$\forall i \in \{1,...n\}, Y_i \mid Z_i \sim \mathcal{P}(e^{\mu_i+Z_j})$
\end{center}
and 
\begin{center}
$\forall (i,j) \in \{1,...,n\}^2$, $ Y_i\mid Z_i$ and $Y_j\mid Z_j$ independant.
\end{center}

We typically consider $N$ independant sampling experiment of $n$ variables each time, for example the case of sampling $n$ species over $N$ sites. In our notations, the exposant allways stands for the sampling experiment, and the underscore stand for the observed variable.\\
So we now have $(Z^j)_{j \in \mathbb{N}}$ latent variables, independant and identically distributed : $\forall j \in \{1,...,N\}$, $Z^j \sim \mathcal{N}(0,\Sigma)$, and :
\begin{center}
$\forall (i,j) \in \{1,...,n\} \times \{1,...,N\}$, $Y^j_i \mid Z^j_i \sim \mathcal{P}(e^{\mu_i+Z^j_i})$.
\end{center}
The advantage of this model is that all the dependance structure is encoded through gaussian variables.
\subsubsection{Extension accounting for covariates and offset}
In multivariate data, the dependency can occur through covariates. As developped in \cite{chiquet2017variational}, we will postulate the existence of a linear regression in  the parameters space. \\
\\
In this case, we consider a vector $X^j \in \mathbb{R}^d$ of covariates (collected for each sampling). Let $M$ be a $d \times n$ Matrix, called the matrix of regression parameters. We note $\mu^i$ the $i^{th}$ column of the matrix M. We can also add one offset parameter per observation $O^j_i$. The offset, can typically be the sampling effort. In this case the Poisson log-normal model is given by :
\begin{center}
$\forall (i,j) \in \{1,...,n\} \times \{1,...,N\}$, $Y^j_i \mid Z^j_i \sim \mathcal{P}(e^{O^j_i + X_j^T \mu^i +Z^j_i})$ 
\end{center}
The set of parameters, that we aim to estimate is now given by $(M,\Sigma) \in \mathcal{M}_{d \times n}(\mathbb{R}) \times \mathcal{M}_{n}(\mathbb{R})$. $M$ is the matrix of regression parameter. Each column of $M$ contain all the regression parameters toward the covariables for one observed variable. We will now use the following notations : $X \in \mathcal{M}_{d \times N} ( \mathbb{R})$ stands for the matrix of covariates. The $j^{th}$ column of $X$ contains the covariates for the $j^{th}$ sampling. $O =(O_{ij}) \in \mathcal{M}_{n \times N} (\mathbb{R})$ stands for the matrix of offsets, we consider that $O_{ij}=O^j_i$. \\
\\
This model enable us to take in account for fixed additional effects. Indeed, it can help to interpret the dependency parameters. For examples, 	Poisson log-normal model can be used to estimate an interaction network from the co-occurancy. With this model, we can hope, with a good choice of the covariables, that the dependencies encoded in the variance-covariance matrix $\Sigma$ will indicate the interraction between species.
\subsection{Properties}
\subsubsection{Density function of the Poisson log-normal distribution}
Following \cite{aitchison1989multivariate}, the Poisson log-normale distribution admits a density function given by :
\begin{center}
$\forall (m_1,...m_n) \in \mathbb{N}^n$, $h_{(\mu,\Sigma)}(m_1,...m_n)=\int_{\mathbb{R}^n} \prod_{i=1}^n f_{e^{\mu_i+z_i}}(m_i) g_{(0,\Sigma)}(z_1,...z_n) \mathrm{d}z_1...\mathrm{d}z_n$
\end{center}
Where $f_{\alpha}$ stands for the density function of a Poisson distribution with parameter $\alpha$, and $ g_{(0,\Sigma)}$ for the density function of a multivariate Normal distribution with parameters $(0,\Sigma)$.\\
\\
Taking in account covariates and offset, give the following distribution function :
\begin{center}
$ \forall X \in \mathbb{R}^d, \forall O \in \mathbb{R}^d, \forall (m_1,...m_n) \in \mathbb{N}^n$, $h_{(M,\Sigma \mid X, O)}(m_1,...m_n)=\int_{\mathbb{R}^n} \prod_{i=1}^n f_{e^{O_i+X^T \mu^i+z_i}}(m_i) g_{(0,\Sigma)}(z_1,...z_n) \mathrm{d}z_1...\mathrm{d}z_n$
\end{center}
\subsubsection{Moments of the Poisson log-normale distribution} Again, following \cite{aitchison1989multivariate}, the moments of the Poisson log-normale distribution can easily be obtained through conditional expectation. Considering the observation $Y^j_i$ is following a Poisson log-normal distribution with  parameters $(M , \Sigma)$ (with $\Sigma= (\sigma_{ij})$), the covariates $X \in \mathcal{M}_{d \times N} ( \mathbb{R})$ and the offset $O \in \mathcal{M}_{n \times N} (\mathbb{R})$ 
\begin{align*}
\mathbb{E}(Y^j_i)& =\mathrm{exp}(O_{ij}+ (X^j)^T\mu^i+\frac{1}{2}\sigma_{ii})\\
\mathbb{V}(Y^j_i)& =\mathrm{exp}(O_{ij}+(X^j)^T \mu^i+\frac{1}{2}\sigma_{ii})+(\mathrm{exp}(O_{ij}+(X^j)^T\mu^i+\frac{1}{2}\sigma_{ii})^2(\mathrm{exp}(\sigma_{ii})-1)\\
\mathrm{Cov}(Y^j_i,Y^j_k)& =\mathrm{exp}(O_{ij}+(X^j)^T \mu^i+\frac{1}{2}\sigma_{ii})\mathrm{exp}(O_{kj}+(X^j)^T \mu_k+\frac{1}{2}\sigma_{kk})(exp(\sigma_{ik}) - 1)
\end{align*}
\subsubsection{Overdispersion}
From the calculation of the moments, we see that for $Y^j_i$ following a Poisson log-normal distribution :
\begin{center}
$\mathbb{E}(Y^j_i) < \mathbb{V}(Y^j_i)$.   
\end{center}
The Poisson log-normal distribution is overdispersed, giving a clue that it can be applyed to a large range of multivariate count data, specially in ecology.
Indeed $\mathrm{Cov}(Y^j_i,Y^j_k)$ and $\sigma_{ik}$ have the same sign.
\subsection{Variational estimation of the parameters}
\textit{\`a travailler PLN/PCA}
\SR{}{Pour ce passage, il faut faire court en rappelant le principe de VEM et en disant qu'on sait le faire pour PLN.}
\section{Composite likelihood estimation}
\subsection{Definition and notation}
\subsubsection{Presentation of the composite likelihood}
One classical method to estimate the parameters of a distribution is to maximize a so called likelihood-function over all the parameters sets. The likelihood-function is typically the density of the modelled distribution taken over all observations :
\begin{center}
$\mathcal{L}_{(Y^1,..Y^N)}(M,\Sigma \mid X,O) = \prod_{j=1}^N h_{(M,\Sigma \mid X,O)}(Y^j)$
\end{center}
When the considered density function is in the exponential family, people often take the log-likelihood, expressed as follow : 
$\mathcal{L}_{(Y^1,..Y^N)}(M,\Sigma \mid X,O) = \sum_{j=1}^N \mathrm{log} (h_{(M,\Sigma \mid X,O)}(Y^j))$.\\
\\ 
In our case, the likelihood function is costly to calculate, because the dependency structure in the data implies to calculate integrals over $\mathbb{R}^n$. Several approached have been proposed in order to symplify the likelyhood function in the case of complex dependencies. We propose to use the composite likelihood approach (see\cite{varin2011overview}, \cite{pedeli2018pairwise} and \cite{varin2008composite}).\\
\\
Composite likelihood is expressed as a weighted product of the marginal or conditional densities.
\begin{definition}\cite{varin2011overview} Let $Y$ be a $n$-dimensional random vector, with density function $f_\theta$ parametrized by a $p$-dimensional unknown parameter $\theta \in \Theta$.\\
Let $\{\mathcal{A}_1,..., \mathcal{A}_k\}$ be a set of marginal or conditional events. We note $\mathcal{L}^k_{(y)} (\theta) \varpropto f_\theta(y)$ the associated (marginal or conditional) likelihood. The composite likelihood is then given by :
\begin{align*}
\mathcal{CL}_{(y)}(\theta) = \prod_{i=1}^k (\mathcal{L}^i_{(y)}(\theta))^{w_i} 
\end{align*}
Where $w_i$ are non negative weights to be choosen.
\end{definition}
The idea behind composite marginal likelihood \cite{varin2008} is to compose low dimensional marginal densities in order to symplify the calculations, but also to capture the dependence between the parameters. \\
If there is no dependance structure, it is sufficient to take the product of the one-dimensional densities. Otherwise, the two-dimensionnal marginal densities are needed at least, to capture the  dependence between the parameters.\\
\\
No theoretical results exist about the loss of efficiency \cite{lele2006sampling}. The idea is to show that maximizing the composite likelihood is a consitent assymptotic estimator of the parameters.

\subsubsection{Composite likelihood for the PLN model}
In our case, by integration, we can show that the two-dimensional marginal density function is given by $ \forall (m_1,m_2) \in \mathbb{N}^2$,$\forall X \in \mathbb{R}^d$, $ \forall O \in \mathbb{R}^d$ :
\begin{center}
 $\sum_{(m_3...m_n) \in \mathbb{N}^{n-2}} h_{(\mu,\Sigma \mid X)}(m_1,...m_n) = \int_{\mathbb{R}^2} f_{e^{O_1+X^T \mu^1+z_1}}(m_1) f_{e^{O_2+X^T \mu^2+z_2}}(m_2) g_{(0,\Sigma^{(12)})}(z_1,z_2)\mathrm{d}z_1 \mathrm{d}z_2$.
\end{center}
with $\Sigma^{(12)}=\begin{pmatrix}
\sigma_{11} & \sigma_{12} \\
\sigma_{12} & \sigma_{22}\\
\end{pmatrix}$\\
With this simple expression of the pairwise density function, we can write the composite pairwise marginal density function given by : $\forall (Y^1,...,Y^N) \in (\mathbb{R}^n)^N$, $\forall O \in \mathcal{M}_{n \times d} (\mathbb{R})$, $\forall X \in \mathcal{M}_{d \times N}(\mathbb{R})$ :
\begin{align}
\mathcal{CL}_{(Y^1,...,Y^N,O)}(M,\Sigma \mid X) = \prod_{j=1}^N \prod_{1 \leq i < k \leq n}  h'_{(M^{(ik)},\Sigma^{(ik)} \mid X^j,O^j)}(Y^j_i,Y^j_k)
\end{align}
With $h'$ the pairwise density function defined as above , $M^{(ik)}$ the matrix constituted of the $i$-th and $k$-th row of M and 
$\Sigma^{(ik)}  = \begin{pmatrix}
\sigma_{ii} & \sigma_{ik}\\
\sigma_{ik} & \sigma_{kk}\\
\end{pmatrix}$
For easier calculation, we take the log composite likelihood given by :
\begin{align*}
\mathcal{CL}_{(Y^1,...,Y^N,O)}(M,\Sigma \mid X) = \sum_{j=1}^N \sum_{1 \leq i < k \leq n}  \mathrm{log}(h'_{(M^{(ik)},\Sigma^{(ik)} \mid X^j,O^j)}(Y^j_i,Y^j_k))
\end{align*}
There exists some theoretical results about the consistency of the method of maximisation of the composite likelihood. In the next section we show that the pariwise log-composite likelihood, is an M-estimator and so all the theory developped for M-estimators \cite{vaart_1998} can be applied.
\subsubsection{M-estimator and composite likelihood}
\begin{definition}
Let $(Y^1,...Y^N) \in \mathcal{X}^N$ be a set of observations. Let $\theta$ be an unknown parameter. \\
$\widehat{\theta}_N(Y^1,...Y^N)$ is called an M-estimator of $\theta$ if it maximize a function :
\begin{align*}
\mathcal{M}_N : \theta \mapsto \frac{1}{N} \sum_{i=1}^N m_\theta(Y^i)
\end{align*}
with, for all $\theta \in \Theta$, $m_\theta : \mathcal{X} \rightarrow \mathbb{R}$ a known function.
\end{definition}
Since the $\frac{1}{N}$ factor in the M-estimator doesn't modify the location of the maximum, we can also define the M-estimator as the parameter miximizing a function $\mathcal{M}_N : \theta \mapsto  \sum_{i=1}^N m_\theta(Y^i)$.\\
\\
In the case of the composite likelihood, for all $\theta \in \Theta$, we set 
\begin{center}
$\begin{array}{ccccc}
m_\theta & : & \mathcal{X} & \to & \mathbb{R} \\
 & & Y^j & \mapsto & \sum_{0 \leq i<k \leq n} \mathrm{log} (h'_{(M^{(ik)},\Sigma^{(ik)}}(Y^j_i,Y^j_k) \\
\end{array}$
\end{center}
The estimator constructed by maximizing the pairwise composite likelihodd is an M-estimator. We will apply the theory developped for M-estimators to prove the consistency and assymptotic normality of the estimator.
\subsection{Consistency}
\subsubsection{Consistency of M-estimators}
\begin{theorem} \label{ThMest} \cite{vaart_1998}
Let $(\mathrm{M}_N)$ be a random sequence of functions in the variable $\theta$ and $M$ a determinist function in the variable $\theta$. If :
\begin{enumerate}
\item $\underset{\theta \in \Theta}{\mathrm{sup}} \mid \mathrm{M}_N(\theta)-\mathrm{M}(\theta) \mid \overset{\mathbb{P}}{\longrightarrow} 0$
\item the maximum $\theta^\ast$ of M is unique.
\end{enumerate}
Then any sequence of estimators $\widehat{\theta}_N$ with $\mathrm{M}_N(\widehat{\theta}_N) \geq \mathrm{M}_N(\theta^\ast)-\circ_p(1)$ converges in probability to $\theta^\ast$.
\end{theorem}
\begin{proof}
Let $M_N$ a random sequence of function, $M$ a determinist function with a unique maximum $\theta^*$ and $\widehat{\theta}_N$ be a sequence of estimators satisfying the conditions of the theorem \ref{ThMest}.

We want to proove that :
\begin{center}
$\forall \epsilon > 0$, $\mathbb{P}(\{\mathrm{d}(\widehat{\theta}_N,\theta*)>\epsilon\})\underset{ N \rightarrow + \infty}{\longrightarrow} 0$
\end{center}
Let consider $\epsilon >0$. By assumption, $\theta ^*$ is the unique maximum of the function $M$, so $\underset{ \theta : \mathrm{d}(\theta,\theta^*)}{\mathrm{sup}}(M(\theta)) < M(\theta^*)$. In particular, there exists $\eta > 0$ so that : $\forall \theta \in \Theta$, $\mathrm{d}(\theta,\theta^*)>\epsilon$, $M(\theta) < M(\theta^*)-\eta$.\\
Consequently, the probabilist event $\{\mathrm{d}(\widehat{\theta}_N,\theta^*)> \epsilon\}$ is included in the event $\{M(\widehat{\theta}_N)<M(\theta^*)-\eta\}$. We will show that the probability of this event converges to zero when $N$ tends to infinity.\\
The assumption of uniform convergence of the sequence of functions $M_N$  to $M$ ensures that $M_N(\theta^*) \overset{\mathbb{P}}{\longrightarrow} M(\theta^*)$. Indeed, by assumption $M_N( \widehat{\theta}_N) \geq M_N(\theta^*)- \circ_\mathbb{P}(1)$, so we deduce that $M_N( \widehat{\theta}_N)+ \circ_\mathbb{P}(1) \geq M(\theta^*)$.
We now deduce that :
\begin{align*}
M(\theta^*)-M(\widehat{\theta}_N) & \leq \widehat{\theta}_N)+ \circ_\mathbb{P}(1)-M(\widehat{\theta}_N)\\
& \leq \underset{\theta}{\mathrm{sup}} \mid M_N-M \mid (\theta) + \circ_\mathbb{P}(1)
\end{align*}
This is sufficient to conclude since for all $\eta > 0$, $\mathbb{P} ( \mid M(\theta^*)-M(\widehat{\theta}_N)\mid > \eta) \underset{ N \rightarrow + \infty} {\longrightarrow} 0$.
\end{proof}
\subsubsection{Consistency of the composite likelihood for the PLN model}
The estimator of the parameters $M$ and $\Sigma$ of a Poisson log-normale model constructed by maximizing the composite likelihood is consistent. We show this result in two steps. We first show that for a Poisson log-normal model in dimesion two, the maximum-likelihood estimator is consistent before generalizing to $n$-dimensional PLN models.

\paragraph{In dimension 2} :
\begin{theorem} \label{Consistence_2}
For each $(i,k) \in \{1...n\}^2$, $i \neq k$, the estimator $(\widehat{M}^{(ik)}_N,\widehat{\Sigma}^{(ik)}_N)$ constructed by maximizing the log-likelihood of the couple $(Y^j_i,Y^j_k)_{j \in \{1...N\}}$, given by $\sum_{i=1}^{N} \mathrm{log}(p_{(M^{(ik)},\Sigma^{ik} \mid X)}(Y^j_i,Y^j_k))$, is a consistent estimator of the correlation coefficients $M^{(ik)}$ and of the variance-covariance matrix $\Sigma^{(ik)}$.
\end{theorem}
\begin{proof}

The idea of this proof is to use the theorem \ref{ThMest} for our composite likelihood. The proof will  follow four steps to verify that the assumptions of the theorem are verified. We will first show the convergence in probability of our M-estimator to a function $\mathrm{M}_{(M^{ik},\Sigma^{ik})}$. Then we proove the existence of a unique maximum for this function $\mathrm{M}$. To do this we need two steps : at first we show that our model is identifiable and then, using the Kullback-divergence, that it has a maximum. Combining this two steps will allow us to conclude the existence of a unique maximum. In the last part, we conclude, using the theroem \ref{ThMest} on the consistence of the estimator. \\
\\
\textbf{Step 1 :} Convergence in probability.\\
Using the large number low, we do have that $\frac{1}{N} \sum_{i=1}^{N} \mathrm{log}(h_{(M^{(ik)},\Sigma^{(ik)} \mid X)}(Y^j_i,Y^j_k))$ converge almost surely to $\mathbb{E}_X[\mathbb{E}_{Y_i Y_k \mid X}[ \mathrm{log}(h_{(M^{(ik)},\Sigma^{(ik)} \mid X)}(Y_i,Y_k))]]$. So we set $\mathrm{M}(M^{(ik)}, \Sigma^{(ik)}) =\mathbb{E}_X[\mathbb{E}_{Y_i Y_k \mid X}[ \mathrm{log}(h_{(M^{(ik)},\Sigma^{(ik)} \mid X)}(Y_i,Y_k))]]$. \\
\\
\textbf{Step 2 :} Identifiability of the model.\\
 We use the general definition of identifiability \cite{rivoirard}, namely that the family of probabilities $\mathcal{P}=\{\mathbb{P}_\theta,\theta \in \Theta\}$ is identifiable if the application $ \theta \mapsto \mathbb{P}_\theta$ is injective.
 In our case $\mathcal{P}=\{(\mathcal{PLN}_X (M,\Sigma))_{X \in \mathcal{X}}; (\mu, \Sigma) \in  \mathcal{M}_{d \times n}(\mathbb{R}) \times \mathcal{M}_n(\mathbb{R})\}$. To show the identifiability of the model we will use the fact that two variables having the same distribution have the same moment. Consider $(M^{(ik)},\Sigma^{(ik)}) \in \mathcal{M}_{d\times 2}(\mathbb{R}) \times \mathcal{M}_2(\mathbb{R})$ and $(M'^{(ik)},\Sigma'^{(ik)}) \in \mathcal{M}_{d\times 2}(\mathbb{R}) \times \mathcal{M}_2(\mathbb{R})$, so that  for every vector of covariables $X \in \mathcal{X}$,$\mathcal{PLN}_X(M^{(ik)},\Sigma^{(ik)})\sim\mathcal{PLN}_X(M'^{(ik)},\Sigma'^{(ik)})$ . We have :
 \begin{equation}
 \mathbb{E}_{(M^{(ik)},\Sigma^{(ik)} \mid X)}[Y_i]=\mathbb{E}_{(M'^{(ik)},\Sigma'^{(ik)} \mid X)} [Y_i] 
 \end{equation}
\begin{equation}
 \mathrm{Var}_{(M^{(ik)},\Sigma^{(ik)} \mid X)}[Y_i]=\mathrm{Var}_{(M'^{(ik)},\Sigma'^{(ik)} \mid X)} [Y_i] \\
\end{equation}
\begin{equation}
 \mathrm{Cov}_{(\mu^{(ik)},\Sigma^{(ik)} \mid X)}[Y_i,Y_k]=\mathrm{Cov}_{(M'^{(ik)},\Sigma'^{(ik)} \mid X)} [Y_i,Y_k]
\end{equation}
Using the formula of this moment we recall in the presentation of the model, we find that $\sigma_{ii} = \sigma'_{ii}$, $\sigma_{ik}=\sigma'_{ik}$ and $X (M^i-M'^i)=0$. The last condition has to be true for any vector of covariable $X$. We deduce from this that,  we expect $\mathcal{X}^\perp = \{0\}$ for the model to be identifable, \textit{ie.} that $\mathrm{rg}(\mathcal{X})=d$.\\
\\
\textbf{Step 3 :} Existence of a unique maximum for the function $\mathrm{M}$.\\
What we consider is nothing else than the log-likelihood of two variables with a Poisson-log-normale distribution. We just shew that the model is identifiable. So there exist one and only one set of parameters $(M^{(ik)}^\ast, \Sigma^{(ik)}^\ast)$ such that $(Y_j,Y_k) \sim \mathcal{PLN}_X(M^{(ik)}^\ast,\Sigma^{(ik)}^\ast)$. We first can show that maximizing the log-likelihood is equivalent to minimize the Kullback divergence. We recall that the Kullback divergence for two distributions of density functions $p_\theta$ and $p_{\theta^\ast}$ is given by $\mathrm{D_{KL}}(p_{\theta^\ast} \parallel p_{\theta})= \int p_{\theta^\ast}(x) \mathrm{log}(\frac{p_{\theta^\ast}(x)}{p_{\theta}(x)}) \mathrm{d}x$. Since we have :
\begin{align*}
\mathbb{E}_X[\mathbb{E}_{(M^{(ik)}^\ast,\Sigma^{(ik)}^\ast \mid X)}&[\mathrm{log}(h_{(M^{(ik)}^\ast,\Sigma^{(ik)}^\ast \mid X)}(Y_i,Y_k)) - \mathrm{log}(h_{(M^{(ik)},\Sigma^{(ik)} \mid X)}(Y_j,Y_k))]] \\
& = \mathbb{E}_X[\mathrm{D_{KL}}(h_{(\mu_{jk}^\ast,\Sigma_{jk}^\ast \mid X)}\parallel h_{(\mu^{(ik)},\Sigma^{(ik)} \mid X)})]\\
& = \mathbb{E}_X[\mathbb{E}_{(M^{(ik)}^\ast,\Sigma^{(ik)}^\ast \mid X)}[\mathrm{log}(h_{(M^{(ik)}^\ast,\Sigma^{(ik)}^\ast \mid X)}(Y_i,Y_k))]] - \mathrm{M}(M^{(ik)},\Sigma^{(ik)})
\end{align*}
 Since the Kullback-divergence is always positive, we see that maximizing the log-likelihood is equivalent to minimize the Kullback-divergence. The Kullback-divergence is only zero if the two distributions are the same. So we see that the M function is maximum for $(M^{(ik)},\Sigma^{(ik)})=(M^{(ik)}^\ast,\Sigma^{(ik)}^\ast)$. So we can conclude that the function M only has one maximum, and this maximum is obtained for the parameters we are interested in.\\
\\
\textbf{Step 4 :} Conclusion.\\
To summurize we have :
\begin{enumerate}
\item By the convergence almost surely :
\begin{center}
 $\underset{(M^{(ik)},\Sigma^{(ik)})}{\mathrm{sup}} \mid \frac{1}{N} \sum_{j=1}^{N} \mathrm{log}(h_{(M^{(ik)},\Sigma^{(ik)} \mid X)}(Y^j_i,Y^j_k)) - \mathbb{E}_X[\mathbb{E}_{Y_i Y_k \mid X}[\mathrm{log}(h_{(M^{(ik)},\Sigma^{(ik)} \mid X)}(Y_i,Y_k))]] \mid \overset{\mathbb{P}}{\longrightarrow} 0 $ 
 \end{center}
\item The function $(M^{(ik)},\Sigma^{(ik)})\mapsto \mathbb{E}_X[\mathbb{E}_{Y_i Y_k \mid X}[\mathrm{log}(h_{(M^{(ik)},\Sigma^{(ik)} \mid X)}(Y_i,Y_k))]] $ only has one maximum for $(M^{ik},\Sigma^{(ik)})=(M^{(ik)}^\ast,\Sigma^{(ik)}^\ast)$, the parameters of the Poisson-log-Normale distribution of the variables $(Y_i,Y_k)$.
\end{enumerate}
Applying theorem \ref{ThMest}, we conclude that the sequence of estimators $(\widehat{M}^{(ik)}_N, \widehat{\Sigma}^{ik}_N)_{N \in \mathbb{N}}$ converges in probability to what we hope to approximate, namely $(M^{(ik)}^\ast, \Sigma^{(ik)}^\ast$).
\end{proof}

\paragraph{In dimension $n$}
\begin{theorem}
The estimator $(\widehat{M}_N,\widehat{\Sigma}_N)$ constructed by maximizing the composite pairwise likelihood for $(Y^j_1,...Y^j_n)_{j \in \{1...N\}}$ is a consistent estimator of the correlation coefficients $M$ and of the matrix of variance-covariance $\Sigma$.
\end{theorem}

\begin{proof}
The proof follows exactly the same path that we did for a couple of variables. We will again use theorem \ref{ThMest} and the results of theorem \ref{Consistence_2}.\\
\\
\textbf{Step 1 :} Convergence in probability.
Again using the large number law we have :\\
 $\frac{1}{N}\sum_{i=1}^{N} \sum_{j<k} log(p_{(M^{(ik)},\Sigma^{(ik)} \mid X)} (Y^j_i,Y^j_k) \overset{\mathbb{P}}{\longrightarrow} \sum_{j<k}\mathbb{E}_X [\mathbb{E}_{Y_i Y_k \mid X}[\mathrm{log}(p_{(M^{(ik)},\Sigma^{(ik)} \mid X)}(Y_i,Y_k))]]$.\\
 We note M the function $\mathrm{M} : (M, \Sigma) \mapsto \sum_{j<k}\mathbb{E}_X [\mathbb{E}_{Y_i Y_k \mid X}[\mathrm{log}(p_{(M^{(ik)},\Sigma^{(ik)} \mid X)}(Y_i,Y_k))]]$.\\
 \\
 \textbf{Step 2 :} Identifiability of the model.\\
The family of model we consider here is $\mathcal{P}=\{ (\mathcal{PLN}_X(\mu,\Sigma))_{X \in \mathcal{X}}; M \in \mathcal{M}_{d \times n}(\mathbb{R}), \Sigma \in \mathcal{M}_n(\mathbb{R})\}$. Using the moment as we did with a Poisson log-normale distribution of a vector of two variables, but this time for n variables is sufficient to show the identifiability of the model.\\
\\
\textbf{Step 3 :} Existence and uniqueness of the maximum for the function M.\\
Again here we have by the identifiability of the model, under the assumption that the data follow a Poisson log-normale distribution, that there exist one and only one set of parameters $(M^\ast, \Sigma^\ast)$ such as given a vector of covariables $X$, $(Y^j) \sim \mathcal{PLN}_X (M^\ast, \Sigma^\ast)$. As we did in the previous section we show that maximizing M is equivalent to minimize the function :\\
$(M, \Sigma) \mapsto \sum_{j<k} \mathbb{E}_X [\mathrm{D_{KL}}(p_{(M^{(ik)}^\ast,\Sigma^{(ik)}^\ast \mid X)} \parallel p_{(M^{(ik)},\Sigma^{(ik)} \mid X)}]$.\\
We have here a finite sum of positives variables, so in order to minimize it, we can minimize each of the terms. So the above function is minimal for $M^\ast=(M^{(i)}^\ast)_{i \in \{1...n\}}$ and $\Sigma^\ast$ defined as follow : the term $(\sigma_{ik})_{i \neq k}$ of $\Sigma^\ast$ is the term $\sigma_{12}$ of $\Sigma^{(ik)}^\ast$  and the term $\sigma_{ii}$ of $\Sigma^\ast$ is the term $\sigma_{11}$ of $\Sigma^{(ik)}^\ast$.\\
Thus we have the existence of a maximum and its uniqueness.\\
\\
\textbf{Step 4 :} Conclusion.\\
Applying theorem \ref{ThMest}, we conclude that the sequence of estimators $(\widehat{M}_N, \widehat{\Sigma}_N)_{N \in \mathbb{N}}$ maximizing the composite likelihood function $(M,\Sigma) \mapsto \mathcal{CL}_{(M,\Sigma) \mid X}((Y^j)_{j \in \{1..N\}})$  converges in probability to what we hope to approximate, namely $(M^\ast, \Sigma^\ast)$, the parameters of our Poisson log-normale model.
\end{proof}
\subsubsection{Estimation of the maximum.}
In order to estimate the maximum of the composite pairwise likelihood we performed an optimization algorithme with the gradients. We calculate the gradients and performed simulations to estimate the parameters.\\
The calculation of the density function of the Poisson log-normale distribution are quiet costly to calculate. Since the optimization require calculation of the density function, we propose to initialise the optimisation algorithme with the results of the VEM for the Poisson log-normal model.
\subsection{Assymptotic normality}
The assymptotic normality is needed to construct tests in order to construct assymptotic confidence sets.
\subsubsection{M-estimator theory}
As in the previous section we define :
\begin{center}
$\Psi_n(\theta)=\frac{1}{N}\sum_{i=1}^N \Psi_\theta(X_i) = \mathbb{P}_N \Psi_\theta$\\
$\Psi(\theta) = \mathrm{P}\Psi_\theta$
\end{center}
\begin{theorem} \cite{vaart_1998} \label{ThAN}
Let $\Theta$ be an euclidian space. For each $\theta \in \Theta$, let $\theta \mapsto \Psi_\theta (x)$ une fonction $\mathcal{C}^2$ with respect to the variable $x$. Suppose that there exists $\theta_0$, such that $\mathrm{P} \Psi_{\theta_0} =0 $ and $\mathrm{P} \parallel \Psi_{\theta_0} \parallel^2 < \infty$. Indeed, we suppose that the matrix $\mathrm{P} \dot{\Psi}_{\theta_0}$ exists and is non-singular. Assume that the second-order partial derivatives are dominated by a fixed integrable function in the variable  $\ddot{\Psi}$ for every $\theta$ in a neiborhood of $\theta_0$. Consider a consistent estimator sequence $\widehat{\theta}_N$ such that $\Psi_N(\widehat{\theta}_N)=0$, the sequence $\sqrt{n}(\widehat{\theta}_N - \theta_0)$ converges in law to $\mathcal{N}(0,(\mathrm{P} \dot{\Psi}_{\theta_0})^{-1} \mathrm{P} \Psi_{\theta_0} \mathrm{P} \Psi_{\theta_0}^T (\mathrm{P} \dot{\Psi}_{\theta_0})^{-1})$.
\end{theorem}
\subsubsection{Assymptotic normality of the composite likelihood for the PLN model}
We want to apply theorem \ref{ThAN} to our problem of M-maximization. In this case the function $\Psi_\theta$ is given by the derivative of the composite likelihood.\\
\\
We did not do the rigourous proof of this result, and we present here some tracks we explored to show this result.
At first, we see that the composite likelihood $(M,\Sigma) \mapsto \mathcal{CL}_{(Y^1,...,Y^N,O)}(M,\Sigma \mid X)$ function is three time continuously differentiable with respect to the variable $(M,\Sigma)$ for every $(Y^1,...,Y^N,O)$, beacause it is just the sum of the composition of three time continuously derivable functions.\\
We shew, for the consistency of the estimator, that there exists an unique maximum $(M^*,\Sigma^*)$ for the composite likelihood. We have to show, in order to apply theorem \ref{ThAN}, that :
\begin{enumerate}
\item all this three derivatives of the composite likelihood admit a first order moment with respect to the observations. 
\item the firs order derivative of the composite likelihood with respect to the parameters $(M,\Sigma)$ admits a first order  moment (with respect to the observation) when taken at the value $(M ^*,\Sigma^*)$.
\item the norm of the first order derivative of the composite likelihood with respect to the parameters $(M,\Sigma)$ admits a first order  moment (with respect to the observation) when taken at the value $(M ^*,\Sigma^*)$. 
\item the second order derivative of the composite likelihood with respect to the parameters $(M,\Sigma)$ admits a first order moment (with respect to the observtions) when taken at the value $(M^*,\Sigma^*)$. The obtained matrix has to be non singular.
\item the expectation (with respect to te observations) of the third order derivative of the $\mathcal{CL}$ is dominated in a neighborhood of $(M^*,\Sigma^*)$ by a function integrable over $\mathbb{R}^n$.
\end{enumerate}
 We give here an alternative expression of the derivative of the composite likelihood with respect to the parameters, in order to estimate easily, and more generally the derivates. Since the composite likelihood is a sum of symilar terms, we only write the calculus for one couple of observations $(Y_i,Y_j)$.
 \begin{align*}
 \mathrm{log}(h_{(M^{(ij)},\Sigma^{(ij)}} (Y_i,Y_j)) =\mathbb{E}_{Z_{(ij)}} [\mathrm{log} (f_{e^{X^TM^{(j)}+Z_j}}(Y_j)) + \mathrm{log}(f_{e^{X^TM^{(i)}+Z_i}} (Y_i) ) ]
 \end{align*}
In this way the only dependance in $\Sigma$ is expressed throught the mean with respect to the variable $Z$ and the only dependance in $M$ is in the density function of the Poisson.\\
In order to derivate withe respect to the parameters, we use the well-known mathematical fact : $\partial_x g(x) = g(x) \partial_x \mathrm{log}(g(x))$. Because we splet the dependency in the variables, we get :
\begin{align*}
\partial_{M^{(ij)}} \mathrm{log}(h_{(M^{(ij)},\Sigma^{(ij)}} (Y_i,Y_j)) = \mathbb{E}_{Z_{(ij)}} [\partial_{(M^{(ij)}} \mathrm{log} (f_{e^{X^TM^{(j)}+Z_j}}(Y_j)) ] + \mathbb{E}_{Z_{(ij)}}[ \partial_{(M^{(ij)}} \mathrm{log}(f_{e^{X^TM^{(i)}+Z_i}} (Y_i) ) ]\\
\partial_{\Sigma^{(ij)}} \mathrm{log}(h_{(M^{(ij)},\Sigma^{(ij)}} (Y_i,Y_j)) = \mathbb{E}_{Z_{(ij)}} [( \mathrm{log} (f_{e^{X^TM^{(j)}+Z_j}}(Y_j)) +  \mathrm{log}(f_{e^{X^TM^{(i)}+Z_i}} (Y_i) ) ) \partial_{\Sigma^{(ij)}} \mathrm{log} (g_{(0,\Sigma^{(ij)})} (Z_{(ij)})) ]
\end{align*}
For the derivatives with respect to the regression parameters, the calculation of the derivative are done in the appendix, when we calculate  the gradients (see \ref{grad}).
We get : 
\begin{align*}
\partial_{M^{(ij)}} \mathrm{log}(h_{(M^{(ij)},\Sigma^{(ij)}} (Y_i,Y_j)) = & \mathbb{E}_{Z_{(ij)}} [ X \frac{Y_j f_{e^{X^TM^{(j)}+Z_j}}(Y_j) - (Y_j+1) f_{e^{X^TM^{(j)}+Z_j}}(Y_j+1) }{f_{e^{X^TM^{(j)}+Z_j}}(Y_j)} ] \\
& + \mathbb{E}_{Z_{(ij)}} [ X \frac{Y_i f_{e^{X^TM^{(i)}+Z_i}}(Y_i) - (Y_i+1) f_{e^{X^TM^{(i)}+Z_i}}(Y_i+1) }{f_{e^{X^TM^{(i)}+Z_i}}(Y_i)} ]
\end{align*} 
Since the Poisson law admits a first order moment, we see that the derivative of the composite likelihood with respect to the matrix of regression parameters taken at the value $(M^*, \Sigma^*)$ admits a first order moment.\\
\\
About the derivatives with respect to $ \Sigma$, note that
\begin{center}
$\mathrm{log} (g_{(0,\Sigma^{(ij)})} (Z_{(ij)})) = -\mathrm{log} (2 \pi \sqrt{\mathrm{det}(\Sigma^{(ij)}}) - \frac{1}{2} Z_{ij}^T \Sigma ^{-1} Z_{ij}$.
\end{center} 
Thanks to the formula given in \cite{petersen2008matrix}, we deduce that :
\begin{align*}
\partial_{\Sigma^{(ij)} } \mathrm{log} (\sqrt{\mathrm{det} \Sigma^{(ij)}}) = - (\Sigma^{(ij)}^{-1})^T\\
\partial_{\Sigma^{(ij)} } (- \frac{1}{2} Z_{ij}^T \Sigma^{-1} Z_{ij} ) = \frac{1}{2} \Sigma^{-T} Z_{ij} Z_{ij}^T \Sigma^{-T}
\end{align*}
Putting it back into the expression of the derivative with respect to $\Sigma ^{(ij)}$, one have :
\begin{align*}
\partial_{\Sigma^{(ij)}} \mathrm{log}(h_{(M^{(ij)},\Sigma^{(ij)}} (Y_i,Y_j)) = & \mathbb{E}_{Z^{(ij)}} [ - (\Sigma^{(ij)}^{-1})^T ( \mathrm{log} (f_{e^{X^TM^{(j)}+Z_j}}(Y_j)) +  \mathrm{log}(f_{e^{X^TM^{(i)}+Z_i}} (Y_i) ) ) ] \\
& + \mathbb{E}_{Z^{(ij)}} [\frac{1}{2} \Sigma^{-T} Z_{ij} Z_{ij}^T \Sigma^{-T} ( \mathrm{log} (f_{e^{X^TM^{(j)}+Z_j}}(Y_j)) +  \mathrm{log}(f_{e^{X^TM^{(i)}+Z_i}} (Y_i) ) )]
\end{align*}
The derivative exists and admits also a first order moment with respect to the observations. This allow us to show that the first item is verified. 
\section{Composite likelihood inference for spatial data}
The typical case of spatial data is that you consider counts on different parcels, and you want to take in account the spatial structure (\textit{ie} near parcels should be highly correlated, whereas distant one should be close to be independant)  in the analysis of the response variable. The Poisson log-normal, ensure to perform a linear model while capturing the dependency between the count variables. We propose here to exploit this feature in order to perform linear models on count data with a spatial structure. 
\subsection{Poisson log-normal model for spatial data}
\subsubsection{One specie, spatial dependency}
In the previous section we suppose, more or less implicitely, that the variables of interest are the the counts (for examples of individuals of different species) and that we have repetitions, eather in time or in space. In the case of the spatial model, we consider the count of only one specie on different parcels. The variables of interest will then be the number of individuals on each parcel. The repetition will then be, for exemple, different series in time.\\
\subsubsection{spatial parametrisation of the PLN}
We introduce the following notations : 
\begin{itemize}
\item $n \in \mathbb{N}$ is the number of parcels. $Y_1...Y_n$ are the counts of the number of individuals on each parcel.
\item In order to capture the spatial dependency in the variance-covariance matrix, we suppose that it is of the form 
\begin{align*}
\Sigma = \sigma^2 \begin{pmatrix}
1 &  e^{-\alpha d_{1,2}} &  e^{-\alpha d_{1,2}} & \ldots &  e^{- \alpha d_{1,n}}  \\
 e^{-\alpha d_{1,2}} & 1 & \ldots & \ldots &  e^{- \alpha d_{2,n}}\\
 \vdots & & \ddots & & \vdots\\
 \vdots & & & \ddots & \vdots \\
 e^{-\alpha d_{1,n}} &  e^{-\alpha d_{2,n}} & \ldots & \ldots & 1 
\end{pmatrix}
\end{align*} 
with $d_{ij}$ the distance between the parcels $i$ and $j$, $\alpha$ the so called spatial range, and $\sigma^2$ the variance.
\end{itemize}
 
 In case of a "spatial" PLN, the number of parameters is reduce, since we only need to estimate $\sigma^2$ and $\alpha$ to capture the whole dependency structure. In a sense, there are more constrain in the matrix of variance and covariance, we need to adapt the calculation we previously did to this special case. 
\subsection{Estimation of the parameters for the model spatial}
As we explained for the general PLN-model, we first tried to calculate the gradient of the composite likelihood, in order to optimize the composite likelihood, using an optimization algorithme based on true caclulations of the gradient. Since we failed to get an explicit expression of the gradient under this constrain, we propose to maximize the likelihood with algorithms that do not need exact expressions of the gradient. At least, since this optimisation methods are very costly, we propose restrain the number of pairs taken in account in the composite likelihood.
\subsubsection{Explicit calculation of the gradient}
In \ref{spatial} we get explicit formula of what we obtained when calculating the derivatives with respect to $\alpha$ and $\sigma^2$ of one term of the composite likelihood in this case, namely $\mathrm{log}(h_{(M^{(ij)},\alpha, \sigma^2)}$. We failed to explicitly calculate this term. In turns, we could propose to estimate this quantity using for example numerical approximations such as Monte-Carlo Markov Chain.\\
\\
An other approach we tried, is to use optimisation algorithms that do not need an explicit expression of the gradient. Different algorithms are implemented in the software \texttt{R} and some of them (we used \texttt{optim}) give an prroximation of the hessian matrix at the optimum. This enabled us to use the assymptotic normality to construct confidence sets.
\subsubsection{Computational optimisation by using a sparse composite likelihood}
Spatial datasets often contain a large number $n$  of parcels, so a large number of variable of interest. The pairwise composite likelihood evaluation requires to the evaluation of $n^2$ Poisson log-normal density function. The evaluation of this function is computationally quite costly, since it implies to estimate an integral over $\mathbb{R}^2$. We propose an approach in order to limit the number of pairs taken in account in the composite likelihood. \\
\\
Let consider $I$ a subset of $\{(i,j) \in \{1...n\}^2\}$. We define $\mathcal{CL}^I = \sum_{(i,j) \in I} \mathrm{log} h_{(M^{(ij)},\alpha,\sigma^2)}(Y_i,Y_j) $.The question is, how to choose an optimal subset $I$.\\
\\
The first constraint wa have, is that each $i \in \{1...n\}$ has to appear at least once in $I$, in order to have access to all the regression parameters.\\
\\
We derive the second constraint from the linear regression of the covariances and the distance between the parcels.
Note $\widehat{\sigma_{ij}}$ an estimator of the parameter of covariance $\sigma_{ij}$.
One have, by a linear model :
\begin{align*}
\mathrm{log}(\overhat{\sigma_{ij}}) = \mathrm{log} ( \sigma^2) - \alpha d_{ij} + E_{ij}
\end{align*}
With $E_{ij}$ a noise due to estimation. We estime this noise to be  independant and identically distributed  for all the considered couples. By a classical results for the linear regression model \cite{daudin1999statistique}, one have :
\begin{center}
$\mathbb{V}(\widehat{\alpha}) = \rho^2 (\frac{1}{n^2} + \frac{\bar{d}^2}{\sum_{(i,j)\in \{1..n\}^2}(d_ij - \bar{d})^2})$
\end{center}
Where $\rho^2$ is the variance of the error. We want to minimize the variance of $\widehat{\alpha}$. So we will try to maximize the quantity $\sum_{(i,j)\in \{1..n\}^2}(d_ij - \bar{d})^2$. This is our second constraint.\\
\\
Choosing $I$ under this two constraints, that is actually determining a maximum spanning tree over the following network : the vertices are the points $ \{i \in \{1...n\} \}$ and the edge between $i$ and $j$ is given by $d_{ij}$. We propose to choose the maximum spanning tree by using the Kruskal algorithm \cite{Kruskal1956}. 

\section{Illustrations}
We performed two main illustration of the theory developped in th previous sections, on one hand we performed simulation to chck our methods, on the other hand, we applied our model on spatial data of ticks-abundance in a french forest.
\subsection{Simulations}
\subsection{Exemple of utilisation on spatial data : Ticks abundance in a french forest}
\section{Conclusion}
\appendix
\section{Appendix}
\subsection{Gradients} \label{grad}
We calculate the gardient with respect to all the parameters : the regression variables, the variances and covariances. The composite-likelihood is a expressed as a sum over the couples of variables. Since the derivatives are the same for each couple, we performed calculation for one couple $(j,k)$ as exemple.
\subsubsection{With respect to the regression parameters $M$.}
Let consider $i \in \{1...n\}$. We want to calculate $\partial_{M^{(i)}}\mathrm{log}( h_{(M^{(jk)},\Sigma^{(jk)} \mid X)}(Y_j,Y_k))$.
\begin{align*}
\partial_{M^{(i)}}\mathrm{log}( h_{(M^{(jk)},\Sigma^{(jk)} \mid X)}(Y_j,Y_k)) &= \frac{\partial_{M^{(i)}}h_{(M^{(jk)},\Sigma^{(jk)} \mid X)}(Y_j,Y_k) }{h_{(M^{(jk)},\Sigma^{(jk)} \mid X)}(Y_j,Y_k)}
\end{align*}

Since :
\begin{align*}
\partial_{M^{(i)}}h_{(M^{(jk)},\Sigma^{(jk)} \mid X)}(Y_j,Y_k) &= \mathbb{E}_{(Z_j,Z_k)}[\partial_{M^{(i)}}h_{(M^{(jk)},\Sigma^{(jk)} \mid X)}(Y_j,Y_k \mid Z_j,Z_k)]\\
&= \mathbb{E}_{(Z_j,Z_k)}[\partial_{M^{(i)}}h_{(M^{(jk)},\Sigma^{(jk)} \mid X)}(Y_j\mid Z_j)h_{(M^{(jk)},\Sigma^{(jk)} \mid X)}(Y_k\mid Z_k)]
\end{align*}
if $i \neq j$ and $i \neq k$, this partial derivatives is equal to 0.
Otherwise, taking for example $i=j$, one have :
\begin{align*}
\partial_{M^{(j)}}h_{(M^{(jk)},\Sigma^{(jk)} \mid X)}(Y_j,Y_k) &= \mathbb{E}_{(Z_j,Z_k)}[\partial_{M^{(j)}}h_{(M^{(jk)},\Sigma^{(jk)} \mid X)}(Y_j\mid Z_j)h_{(M^{(jk)},\Sigma^{(jk)} \mid X)}(Y_k\mid Z_k)]\\
&= \mathbb{E}_{(Z_j,Z_k)}[X(Y_j h_{(M^{(jk)},\Sigma^{(jk)} \mid X)} (Y_j \mid Z_j)-(Y_{j}+1) h_{(M^{(jk)},\Sigma^{(jk)} \mid X)}(Y_j+1\mid Z_j))h_{(M^{(jk)},\Sigma^{(jk)} \mid X)}(Y_k\mid Z_k)]\\
&= X(Y_j h_{(M^{(jk)},\Sigma^{(jk)} \mid X)}(Y_j,Y_k)-(Y_j+1)h_{(M^{(jk)},\Sigma^{(jk)} \mid X)}(Y_j+1,Y_k))
\end{align*}
Where the second equality comes from the fact that $Y_j \mid Z_j$ is following a Poisson-distribution. We have :
\begin{align*}
\partial_{M^{(j)}} h_{M^{(j)},\Sigma^{(jj)} \mid X}(Y_j \mid Z_j) &= \partial_{M^{(j)}} (\mathrm{e}^{Y_j (X^T M^{(j)} + Z_j)} \mathrm{exp}(-e^{X^T M^{(j)} + Z_j}) \frac{1}{Y_j !} )\\
&= X Y_j\frac{\mathrm{e}^{Y_j (X^T M^{(j)} + Z_j)} \mathrm{exp}(-e^{X^T M^{(j)} + Z_j})}{Y_j !} - X \frac{\mathrm{e}^{(Y_j+1) (X^T M^{(j)} + Z_j)} \mathrm{exp}(-e^{X^T M^{(j)} + Z_j})}{Y_j !}\\
&= X(Y_j h_{(M^{(j)},\Sigma^{(jj)} \mid X)} (Y_j \mid Z_j)-(Y_{j}+1) h_{(M^{(j)},\Sigma^{(jj)} \mid X)}(Y_j+1\mid Z_j))
\end{align*}
So we conclude that
\begin{align*}
\partial_{M^{(i)}}\mathrm{log}( h_{(M^{(jk)},\Sigma^{(jk)} \mid X)}(Y_j,Y_k)) &= X(Y_j -(Y_j+1)\frac{H_{(M^{(jk)},\Sigma^{(jk)} \mid X)}(Y_j+1,Y_k)}{h_{(M^{(jk)},\Sigma^{(jk)} \mid X)}(Y_j,Y_k))})
\end{align*}
\subsection{With respect to the matrix of variance - covariance.}
The terms of the variance-covariance matrix are only present in the density function of the latent multivariate gaussian distribution. In order to calculate the derivatives we will in both cases follow the same path :
\begin{enumerate}
\item We calculate the derivatives of the density function of the bivariate normale distribution with respect to the parameter of interest. By derivating under the $\int$ we deduce the derivative of the Poisson-log-normale density function.
\item We integrate by part to have a nice expression of this derivative.
\end{enumerate}
We recall here that the density function of the Poisson log-Normal distribution with regression parameters $M$ and matrix of variance-covariance $\Sigma$, given a vector of covariables $X$ is given by :
\begin{equation*}
h_{(M^{(jk)},\Sigma^{(jk)} \mid X)}(Y_j,Y_k)=\int_{\mathbb{R}^2} f_{e^{X^T M^{(jk)}+z_j}}(Y_j) f_{e^{X^T M^{(jk)}+z_k}}(Y_k) g_{(0,\Sigma^{(jk)})}(z_j,z_k) \mathrm{d}z_j \mathrm{d}z_k
\end{equation*}
With $g_{(0,\Sigma^{(jk)})}$ the density function of the bivariate normale distribution of mean $0$ and variance-covariance $\Sigma^{(jk)}$ and $f_{\beta}$ the density function of a Poisson distribution with parameters $\beta$.
We finally recall that : $h_{(0,\Sigma^{(jk)})}(z_1,z_2)=\frac{1}{2 \pi \sqrt{\mid \sigma_{jj} \sigma_{kk}-\sigma_{jk}^2}\mid} \mathrm{e}^{-\frac{1}{2} \frac{\sigma_{jj} z_1^2 + \sigma_{kk} z_2^2 -2 \sigma_{jk} z_1 z_2}{\sigma_{jj} \sigma_{kk}-\sigma_{jk}^2}}$.

\subsubsection{wrt $\sigma_{jj}$ and $\sigma_{kk}$}
\begin{align*}
\partial_{\sigma_{jj}} g_{(0,\Sigma^{(jk)})} (z_1,z_2) &= \frac{1}{2} [ \frac{-\sigma_{kk}}{\sigma_{jj} \sigma_{kk}-\sigma_{jk}^2} +  \frac{(\sigma_{jk} z_1 - \sigma_{kk} z_2)^2}{(\sigma_{jj} \sigma_{kk}-\sigma_{jk}^2)^2}  ]g_{(0,\Sigma_{jk})}(z_1,z_2)
\end{align*}

Since the bivariate normale distribution admit second order moments, we can apply  the thoerem of derivation under $\int$ and deduce that :
\begin{align*}
\partial_{\sigma_{jj}} h_{(M^{(jk)},\Sigma^{(jk)} \mid X)}(Y_j,Y_k) &=\int_{\mathbb{R}^2} f_{e^{X^T M^{(j)}+z_1}}(Y_j) h_{e^{X^T M^{(k)}+z_2}}(Y_k)\frac{1}{2} [ \frac{-\sigma_{kk}}{\sigma_{jj} \sigma_{kk}-\sigma_{jk}^2} +  \frac{(\sigma_{jk} z_1 - \sigma_{kk} z_2)^2}{(\sigma_{jj} \sigma_{kk}-\sigma_{jk}^2)^2}  ] g_{(0,\Sigma^{(jk)})}(z_j,z_k) \mathrm{d}z_j \mathrm{d}z_k
\end{align*}

\noindent We will now try to find a nicer expression of this derivative. One can remark that $\partial_{z_2} (-\frac{1}{2} \frac{\sigma_{jj} z_1^2 + \sigma_{kk} z_2^2 -2 \sigma_{jk} z_1 z_2}{\sigma_{jj} \sigma_{kk}-\sigma_{jk}^2}) = \frac{\sigma_{jk} z_1 - \sigma_{kk} z_2}{\sigma_{jj} \sigma_{kk}-\sigma_{jk}^2}$.

This gives the clue to integrate by part with respect to $z_2$. We then have :
\begin{align*}
\partial_{\sigma_{jj}} h_{(M^{(jk)},\Sigma^{(jk)} \mid X)}(Y_j,Y_k) =& \int_{\mathbb{R}} [  h_{e^{X^T M^{(j)}+z_1}}(Y_j) h_{e^{X^T M^{(k)}+z_2}}(Y_k) \frac{\sigma_{jk} z_1 - \sigma_{kk} z_2}{\sigma_{jj} \sigma_{kk}-\sigma_{jk}^2} g_{(0,\Sigma_{jk})}(z_1,z_2) ]_{- \infty}^{+\infty} \mathrm{d}z_1\\
 &- \int_{\mathbb{R}^2} Y_k f_{e^{X^T M^{(j)}+z_1}}(Y_j) f_{e^{X^T M^{(k)}+z_2}}(Y_k) \frac{\sigma_{jk} z_1 - \sigma_{kk} z_2}{\sigma_{jj} \sigma_{kk}-\sigma_{jk}^2} g_{(0,\Sigma_{jk})}(z_1,z_2) \mathrm{d}z_1 \mathrm{d}z_2\\
 &- \int_{\mathbb{R}^2} (Y_k+1) f_{e^{X^T  M^{(j)}+z_1}}(Y_j)  f_{e^{X^T M^{(k)}+z_2}}(Y_k+1) \frac{\sigma_{jk} z_1 - \sigma_{kk} z_2}{\sigma_{jj} \sigma_{kk}-\sigma_{jk}^2} h_{(0,\Sigma_{jk})}(z_1,z_2) \mathrm{d}z_1 \mathrm{d}z_2
\end{align*}
The derivatives of the density function of the Poisson distribution is obtained as we did in the previous section.\\
The first rht of this equality being null, we can again integrate by part wrt $z_2$ and we get :
\begin{align*}
\partial_{\sigma_{jj}} p_{(\mu_{jk},\Sigma_{jk} \mid X)}(y_j,y_k) &= y_k^2 p_{(\mu_{jk},\Sigma_{jk} \mid X)}(y_j,y_k)\\
& - ((y_k+1)^2 + y_k (y_k+1) ) p_{(\mu_{jk},\Sigma_{jk} \mid X)}(y_j,y_k+1) \\
& + (y_k+1)(y_k+2) p_{(\mu_{jk},\Sigma_{jk} \mid X)}(y_j,y_k+2)
\end{align*} 
We are calculating this gradients in order to approximate the composite log likelihood so we are in fact interested in $\partial_{\sigma_{jj}} \mathrm{log}p_{(\mu_{jk},\Sigma_{jk} \mid X)}(y_j,y_k)$, which we now can easily calculate.
\begin{align*}
\partial_{\sigma_{jj}} p_{(\mu_{jk},\Sigma_{jk} \mid X)}(y_j,y_k) &= y_k^2 - ((y_k+1)^2 + y_k (y_k+1) ) \frac{p_{(\mu_{jk},\Sigma_{jk} \mid X)}(y_j,y_k+1) } {p_{(\mu_{jk},\Sigma_{jk} \mid X)}(y_j,y_k)}\\
& + (y_k+1)(y_k+2)\frac{ p_{(\mu_{jk},\Sigma_{jk} \mid X)}(y_j,y_k+2)}{p_{(\mu_{jk},\Sigma_{jk} \mid X)}(y_j,y_k)}
\end{align*}
Symetrically we have wrt $\sigma_{kk}$ :
\begin{align*}
\partial_{\sigma_{kk}} p_{(\mu_{jk},\Sigma_{jk} \mid X)}(y_j,y_k) &= y_j^2 - ((y_j+1)^2 + y_j (y_j+1) ) \frac{p_{(\mu_{jk},\Sigma_{jk} \mid X)}(y_j+1,y_k) } {p_{(\mu_{jk},\Sigma_{jk} \mid X)}(y_j,y_k)}\\
& + (y_j+1)(y_j+2)\frac{ p_{(\mu_{jk},\Sigma_{jk} \mid X)}(y_j+2,y_k)}{p_{(\mu_{jk},\Sigma_{jk} \mid X)}(y_j,y_k)}
\end{align*}

\subsubsection{wrt $\sigma_{jk}$ :}
\begin{align*}
\partial_{\sigma_{jk}}h_{(0,\Sigma)}(z_1,z_2) &= [\frac{\sigma_{jk}}{\sigma_{jj} \sigma_{kk} - \sigma_{jk}^2} + \frac{-\sigma_{jk} ( \sigma_{jj} z_1^2 + \sigma_{kk} z_2^2 ) + \sigma_{jk} z_1 z_2 + \sigma_{jj} \sigma_{kk} z_1 z_2}{(\sigma_{jj} \sigma_{kk} - \sigma_{jk}^2)^2}] h_{(0,\Sigma)}(z_1,z_2)
\end{align*}
Again we can apply the theorem of derivating under the $\int$. We deduce that :
\begin{align*}
&\partial_{\sigma_{jk}}p_{(\mu_{jk},\Sigma_{jk} \mid X)}(y_j,y_k) =
\int_{\mathbb{R}^2} p_{e^{X\mu_{jk}+z_1}}(y_j) p_{e^{X\mu_{jk}+z_2}}(y_k)\\
&[   \frac{\sigma_{jk}}{\sigma_{jj} \sigma_{kk} - \sigma_{jk}^2} + \frac{-\sigma_{jk} ( \sigma_{jj} z_1^2 + \sigma_{kk} z_2^2 ) + \sigma_{jk} z_1 z_2 + \sigma_{jj} \sigma_{kk} z_1 z_2}{(\sigma_{jj} \sigma_{kk} - \sigma_{jk}^2)^2}] h_{(0,\Sigma_{jk})}(z_j,z_k) \mathrm{d}z_j \mathrm{d}z_k
\end{align*}

One can note that :
\begin{align*}
\partial_{z_1} ( - \frac{1}{2} \frac{\sigma_{jj} z_1^2 + \sigma_{kk} z_2^2 - 2 \sigma_{jk} z_1 z_2 }{\sigma_{jj} \sigma_{kk}-  \sigma_{jk}^2} ) & \partial_{z_2} ( - \frac{1}{2} \frac{\sigma_{jj} z_1^2 + \sigma_{kk} z_2^2 - 2 \sigma_{jk} z_1 z_2 }{\sigma_{jj} \sigma_{kk}-  \sigma_{jk}^2} ) =\\
&  \frac{-\sigma_{jk} ( \sigma_{jj} z_1^2 + \sigma_{kk} z_2^2 ) + \sigma_{jk} z_1 z_2 + \sigma_{jj} \sigma_{kk} z_1 z_2}{(\sigma_{jj} \sigma_{kk} - \sigma_{jk}^2)^2}
\end{align*}
Which gives a clue that we could do again tow part integrations : one towards $z_1$ and one towards $z_2$. \\
We will do a first integration by part with respect to $z_1$. We note that : $ \partial_{z_1} ( \frac{\sigma_{jk}z_1-\sigma_{kk}z_2}{\sigma_{jj} \sigma_{kk}- \sigma_{jk}^2} h_{(0,\Sigma_{jk})}(z_1,z_2) = [\frac{\sigma_{jk}}{\sigma_{jj} \sigma_{kk} - \sigma_{jk}^2} + \frac{-\sigma_{jk} ( \sigma_{jj} z_1^2 + \sigma_{kk} z_2^2 ) + \sigma_{jk} z_1 z_2 + \sigma_{jj} \sigma_{kk} z_1 z_2}{(\sigma_{jj} \sigma_{kk} - \sigma_{jk}^2)^2}] h_{(0,\Sigma)}(z_1,z_2)$. And taking the derivative of the poisson density with respect to its parameters allows us to conclude that :
\begin{align*}
&\partial_{\sigma_{jk}}p_{(\mu_{jk},\Sigma_{jk} \mid X)}(y_j,y_k) = \int_{\mathbb{R}} [ p_{e^{X\mu_{jk}+z_1}}(y_j) p_{e^{X\mu_{jk}+z_2}}(y_k) \frac{\sigma_{jk}z_1-\sigma_{kk}z_2}{\sigma_{jj} \sigma_{kk}- \sigma_{jk}^2} h_{(0,\Sigma_{jk})}(z_1,z_2)]_{- \infty}^{+ \infty} \mathrm{d}z_2\\
& - \int_{\mathbb{R}^2} (y_j p_{e^{X\mu_{jk}+z_1}}(y_j)- (y_j+1) p_{e^{X\mu_{jk}+z_1}}(y_j+1)) p_{e^{X\mu_{jk}+z_2}}(y_k) \frac{\sigma_{jk} z_1 - \sigma_{kk} z_2}{\sigma_{jj} \sigma_{kk} - \sigma_{jk}^2} h_{(0,\Sigma_{jk})}(z_1,z_2) \mathrm{d}z_1  \mathrm{d}z_2
\end{align*}
The first right-hand term of the equality being equal to zero, we have, integrating by part again but this time wrt $z_2$, we have :
\begin{align*}
&\partial_{\sigma_{jk}}p_{(\mu_{jk},\Sigma_{jk} \mid X)}(y_j,y_k) =\int_{\mathbb{R}^2} (y_j p_{e^{X\mu_{jk}+z_1}}(y_j)- (y_j+1) p_{e^{X\mu_{jk}+z_1}}(y_j+1))\\
& (y_k p_{e^{X\mu_{jk}+z_2}}(y_k)-(y_k + 1) p_{e^{X\mu_{jk}+z_2}}(y_k+1))  h_{(0,\Sigma_{jk})}(z_1,z_2) \mathrm{d}z_1  \mathrm{d}z_2
\end{align*}
What we can also write :
\begin{align*}
\partial_{\sigma_{jk}}p_{(\mu_{jk},\Sigma_{jk} \mid X)}(y_j,y_k) = &y_k y_j p_{(\mu_{jk},\Sigma_{jk} \mid X)}(y_j,y_k)-(y_k+1)y_j p_{(\mu_{jk},\Sigma_{jk} \mid X)}(y_j,y_k+1)\\
& -(y_j+1)y_k p_{(\mu_{jk},\Sigma_{jk} \mid X)}(y_j,y_k)+ (y_k+1) (y_j+1)p_{(\mu_{jk},\Sigma_{jk} \mid X)}(y_j+1,y_k+1))
\end{align*}
Again we are interested in $\partial_{\sigma_{jk}}\mathrm{log}p_{(\mu_{jk},\Sigma_{jk} \mid X)}(y_j,y_k)$ which we can now easily calculate by dividing the above quantity by $p_{(\mu_{jk},\Sigma_{jk} \mid X)}(y_j,y_k)$.

\subsection{Hessian}
\subsection{Spatial gradient and hessian} \label{spatial}


\bibliographystyle{plain}
\bibliography{Biblio.bib}
\end{document}