\documentclass[11pt, a4paper]{article}

\usepackage[utf8]{inputenc} 	% L'encodage. Ca peut être à remplacer par \usepackage[utf8]{inputenc}
\usepackage[T1]{fontenc}
\def\ques{\noindent{\bf Question : }}
\usepackage[english]{babel}
\usepackage[top=2cm, bottom=2cm, left=2cm, right=2cm]{geometry}		% Les marges
%\usepackage{listings}		% Pour mettre du code dans le fichier, il suffira ensuite d'appeler \begin{lstlisting}
%\lstset{
%language=Python,        	% choix du langage : C, Python, R, PHP...
%basicstyle=\footnotesize\color{mygray},       % taille de la police du code
%numbers=left,                   % placer les num�ros de lignes à droite (right) ou à gauche (left)
%numberstyle=\footnotesize,        % taille de la police des num�ros
%numbersep=7pt,                  % distance entre le code et sa num�rotation
%stepnumber=1,
%tabsize=4
%}
\usepackage{lscape}
\usepackage{amsthm}
\newtheorem{theorem}{Theorem}
\setcounter{theorem}{0}
\usepackage{comment} 
\usepackage{float}
%\usepackage{capt-of}
\usepackage{graphicx}	% Si besoin de bosser sur des images. Si besoin, aller sur le SDZ.
\usepackage{color}
\definecolor{mygray}{rgb}{0.2,0.2,0.2}
\usepackage{xcolor}
\usepackage{multicol}
% Les packages pour mettre des expressions un peu math�matis�es.
\usepackage{amsmath}
\usepackage{amssymb}
\usepackage{amsthm}
\usepackage{mathrsfs}
%\usepackage{mathtools}
%\usepackage{dsfont}
\newcommand{\horrule}[1]{\rule{\linewidth}{#1}}
\newcommand{\SR}[2]{\textcolor{gray}{#1}\textcolor{blue}{#2}}


\newtheorem{question}{Q}
\title{	
Composite likelihood inference of the parameters of a Poisson log-normale distribution
}

\date{\today}
\author{Elsa TEULIERE}

\graphicspath{ {./figures/} }
 

\begin{document}
\maketitle
\vspace{1cm}
\section*{Introduction}
\section{Poisson log-normal model}
\subsection{Definitions and notations}
\subsubsection{The Model}
\textit{latent gaussian variables, observations law conditionnal to the latent variables,independance}
\subsubsection{Extension}
\textit{regression parameters,offset}
\subsubsection{Interpretation/advantages}
\textit{dependency in the gaussian latent variables,.... }
\subsection{Properties}
\subsubsection{Density function and moments}
\textit{AiH89-Biometrika}
\subsubsection{Overdispersion}
\subsection{Variational estimation of the parameters}
\textit{\`a travailler PLN/PCA}
\SR{}{Pour ce passage, il faut faire court en rappelant le principe de VEM et en disant qu'on sait le faire pour PLN.}
\section{Composite likelihood estimation}
\subsection{Definition and notation}
\subsubsection{Presentation of the composite likelihood}
\textit{VRF11-StatSinica}
\subsubsection{Composite likelihood for the PLN model}
\textit{Expression, definition of the $\Sigma_{ij},\mu_{ij}$, \'equivalence des couples}
\subsection{Consistency}
\subsubsection{M-estimator theory}
\textit{chap.5 Assymptotic Statistics van der Vaart}
\subsubsection{Consistency of the composite likelihood for the PLN model}
\textit{proof}
\subsection{Assymptotic normality}
\subsubsection{M-estimator theory}
\textit{chap.5 Assymptotic Statistics van der Vaart}
\subsubsection{Assymptotic normality of the composite likelihood for the PLN model}
\textit{proof}
\subsubsection{Construction of assymptotic statistic tests (?)}
\textit{Peut-etre trop scolaire...} \SR{}{Je suis d'accord. Il suffit de dire que la normalite asymptotique nous donne les intervalles de confiance (asymptotiques) et les tests (asymptotiques)}

\section{Composite likelihood inference for spatial data}
\subsection{Poisson log-normal model for spatial data}
\subsubsection{One specie\SR{}{s} \SR{(meme au singulier)}{}, spatial dependency}
\subsubsection{spatial parametrisation of the PLN}
\subsection{Computational optimisation by taking a sparce CL}

\section{Illustrations}
\subsection{Simulations}
\subsection{'Tiques' data}

\appendix
\section{Appendix}
\subsection{Gradients}
\subsection{Hessian}
\subsection{Spatial gradient and hessian}


\end{document}