\documentclass[11pt, a4paper]{article}

\usepackage[utf8]{inputenc} 	% L'encodage. Ca peut être à remplacer par \usepackage[utf8]{inputenc}
\usepackage[T1]{fontenc}
\def\ques{\noindent{\bf Question : }}
\usepackage[english]{babel}
\usepackage[top=2cm, bottom=2cm, left=2cm, right=2cm]{geometry}		% Les marges
%\usepackage{listings}		% Pour mettre du code dans le fichier, il suffira ensuite d'appeler \begin{lstlisting}
%\lstset{
%language=Python,        	% choix du langage : C, Python, R, PHP...
%basicstyle=\footnotesize\color{mygray},       % taille de la police du code
%numbers=left,                   % placer les num�ros de lignes à droite (right) ou à gauche (left)
%numberstyle=\footnotesize,        % taille de la police des num�ros
%numbersep=7pt,                  % distance entre le code et sa num�rotation
%stepnumber=1,
%tabsize=4
%}
\usepackage{lscape}
\usepackage{amsthm}
\newtheorem{theorem}{Theorem}
\setcounter{theorem}{0}
\usepackage{comment} 
\usepackage{float}
%\usepackage{capt-of}
\usepackage{graphicx}	% Si besoin de bosser sur des images. Si besoin, aller sur le SDZ.
\usepackage{color}
\definecolor{mygray}{rgb}{0.2,0.2,0.2}
\usepackage{xcolor}
\usepackage{multicol}
% Les packages pour mettre des expressions un peu math�matis�es.
\usepackage{amsmath}
\usepackage{amssymb}
\usepackage{amsthm}
\usepackage{mathrsfs}
%\usepackage{mathtools}
%\usepackage{dsfont}
\newcommand{\horrule}[1]{\rule{\linewidth}{#1}}
\newcommand{\SR}[2]{\textcolor{gray}{#1}\textcolor{blue}{#2}}


\newtheorem{question}{Q}
\title{	
Composite likelihood inference of the parameters of a Poisson log-normale distribution
}

\date{\today}
\author{Elsa TEULIERE, \SR{St\'ephane ROBIN}{}}

\graphicspath{ {./figures/} }
 

\begin{document}
\maketitle
\vspace{1cm}
\section*{Introduction}
\section{Poisson log-normal model}
\subsection{Definitions and notations}
\subsection{Properties}
\subsubsection{Moments}
\subsubsection{Overdispersion}
\subsection{Variational estimation of the parameters}

\section{Composite likelihood estimation}
\subsection{Definition and notation}
\subsubsection{Presentation of the composite likelihood}
\subsubsection{Composite likelihood for the PLN model}
\subsection{Consistency}
\subsection{Assymptotic normality}
\SR{
\subsection{Calculations and simulations}
}{}

\SR{}{
\section{Composite likelihood inference for spatial data}
\subsection{Poisson log-normal model for spatial data}
\subsection{Choosing the pairs (?)}
}

\SR{
\section{Applications}
\subsection{-}
\subsection{Spatial data}
}{}

\SR{}{
\section{Illustrations}
\subsection{Simulations}
\subsection{'Tiques' data}
}

\appendix
\SR{}{
\section{Appendix}
\subsection{Derivatives}
\subsection{...}
}

\end{document}